\subsection {Vocabolario}

\begin{center}

\rowcolors{2}{orange!20}{orange!5}%colori alternati


\begin{longtable}
{|m{3.5cm}|>{\raggedright}m{8cm}|l|}%Opzioni per formato tabella
\hline %riga in cima alla colonna    
\rowcolor{orange!50}%settare il colore della colonna principale 

\Large\textbf{Voce} &\Large\centering\textbf{Definizione} &\Large\textbf{Sinonimo} \n%voci della prima colonna
\endhead
    Utente& Persona non autenticata che usufruisce della piattaforma &
\n  Nome Utente & Stringa alfanumerica che identifica il nome di un Utente&Username
\n  Chiave di accesso & Stringa alfanumerica scelta da un utente, da tenere segreta&Password
\n  TOTP & Time-based One-Time Password, codice numerico generato dal sistema, utilizzato per l'autenticazione a due fattori&
\n  Registrazione& Funzione di iscrizione alla piattaforma&
\n  Autenticazione& Metodo di accesso al servizio, basato su Username, Password e TOTP& Log in
\n  Utente autenticato& Utente che ha effettuato l'autenticazione&
\n  Amministratore & Addetto alla manutenzione della piattaforma & Admin
\n  Codemonkey& Utente registrato o autenticato come programmatore &
\n  Cliente& Azienda o un semplice privato interessato a utilizzare servizi offerti dalle Codemonkey &
\n  Profilo Completo & Il profilo di una Codemonkey è completo se sono presenti tutte le informazioni richieste dalla piattaforma &
\n  Lavoro & Contratto tra una Codemonkey e un Cliente per lo sviluppo di un progetto software. Può essere \textit{proposto} da un Cliente ed \textit{accettato} o \textit{rifiutato} da una Codemonkey. Se viene accettato, può essere \textit{interrotto} dalla Codemonkey o \textit{terminato} dal Cliente & Progetto, Collaborazione
\n  Tag & Parole chiave che permettono di trovare le Codemonkey, saranno quindi i linguaggi di programmazione e le applicazioni che sono in grado di utilizzare&
\n  Ricerca & Il modo in cui si possono trovare le Codemonkey, si basa sui Tag & Feed
\n  Valutazione& Valore espresso in stelle da 1 a 5 per quantificare le prestazioni di una Codemonkey &
\n  Valutazione Filtrata & Valutazione media di una Codemonkey in base ai Tag impostati &
\n  Recensione & Commento scritto da un Cliente per valutare le prestazioni di una Codemonkey &
\n  (Cliente) Limitato & Cliente che non ha diritto di lasciare Recensioni &
\n  (Cliente o Codemonkey) Sospeso & Utente che non ha diritto di avviare o accettare Lavori &
\n  (Cliente o Codemonkey) Bloccato & Utente che non ha diritto di accedere alla piattaforma & Ban
\n 
\end{longtable}
\label{tab:monkeytable:vocabolario2}
\end{center}

