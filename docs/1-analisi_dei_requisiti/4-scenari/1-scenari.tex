\newcommand{\tableCyan}{%righe con colori misti inseriti dentro
    \n
    \rowcolor{tableCyan!10}
    \cellcolor{tableCyan!22}
}

\newcommand{\ntableCyan}{%righe con colori misti inseriti dentro
    \n
    \rowcolor{tableCyan!5}
    \cellcolor{tableCyan!12}
}

%%%%%%%%%%%%%%%%%%%%%%%%%%%%

\begin{tabularx}{\textwidth}
    {|m{3cm}|X|}
    \hline \rowcolor{tableCyan!40}
    \large\centering\textbf{Titolo}     & \large \centering\textbf{Registrazione}
    \tableCyan      Descrizione         & Sezione che consente la Registrazione di un Utente
    \ntableCyan     Attori              & Utente
    \tableCyan      Relazioni           &
    \ntableCyan     Precondizioni       &
    \tableCyan      Postcondizioni      & L'Utente viene registrato nel sistema come Codemonkey o come Cliente
    \ntableCyan     Scenario Principale &
    \begin{enumerate}
        \item L'Utente inserisce le credenziali (Username, Email, Password) per la creazione di un Account
        \item L'Utente specifica se vuole essere registrato come Cliente o Codemonkey
        \item L'Utente conferma di volersi registrare
    \end{enumerate}
    \tableCyan      Scenari Alternativi & Se lo Username é giá stato scelto o l'Email immessa é collegata a un Account giá registrato non sará possibile procedere alla Registrazione dell'Utente
    \ntableCyan     Requisiti NF        & Sia nel caso in cui venga inserito un Username giá esistente, sia nel caso venga inserita una Email associata ad un Account esistente, dovrá essere segnalato all'Utente che non é stato possibile effettuare la Registrazione spiegandone la motivazione
    \tableCyan      Punti Aperti        & Verrá abilitata l'autenticazione a 2 fattori giá in fase di Registrazione?\newline
    Bisognerá assicurarsi che l'Utente non si sbagli a digitare la Password, ci sarà un campo di conferma Password?
    \n
\end{tabularx}

\newpage

\begin{tabularx}{\textwidth}
    {|>{\arraybackslash}m{3cm}|>{\arraybackslash}X|}

    \hline \rowcolor{tableCyan!37}
    \large\centering\textbf{Titolo}     & \large\centering\textbf{Autenticazione}
    \tableCyan      Descrizione         & Sezione dedicata all'Autenticazione di un Utente
    \ntableCyan     Attori              & Utente
    \tableCyan      Relazioni           & Gestione Profilo, Proponi Lavoro, Fornisci recensione, Area personale, Gestione Utenti  
    \ntableCyan     Precondizioni       & L'Utente deve essere giá registrato all'interno del sistema oppure deve essere un Amministratore.
    \tableCyan      Postcondizioni      & L'Utente sarà Utente Autenticato oppure Amministratore.
    \ntableCyan     Scenario Principale &
    \begin{enumerate}
        \item L'Utente fornisce le credenziali di accesso al sistema
        \item L'Utente conferma di voler entrare con le credenziali immesse
        \item Nel caso in cui le credenziali siano corrette, Codemonkey o Cliente saranno reindirizzati alla Gestione del Profilo, invece l'Amministratore sará reindirizzato verso la Gestione del Sistema
    \end{enumerate}
    \tableCyan      Scenari Alternativi & Nel caso che i dati forniti per l'autenticazione siano incorretti verrá visualizzato un messaggio di errore\newline Nel caso l'utente sia stato bloccato da un Amministratore verrá fornito un messaggio di errore.
    \ntableCyan     Requisiti NF        & Sará sempre presente un bottone per inoltrare una richiesta di recupero Password all'Amministratore.
    \tableCyan      Punti Aperti        & 
    \n
\end{tabularx}

%%%%%%%%%%%%%%%%%%%%%%%%%%%%

\begin{tabularx}{\textwidth}
    {|>{\arraybackslash}m{3cm}|>{\arraybackslash}X|}

    \hline \rowcolor{tableCyan!37}
    \large\centering\textbf{Titolo}     & \large\centering\textbf{Gestione Profilo}
    \tableCyan      Descrizione         & Sezione dedicata alla gestione dei dati di un Account
    \ntableCyan     Attori              & Cliente, Codemonkey
    \tableCyan      Relazioni           & Autenticazione
    \ntableCyan     Precondizioni       & La Codemonkey o il Cliente devono aver effettuato l'Autenticazione
    \tableCyan      Postcondizioni      &
    \ntableCyan     Scenario Principale &
    \begin{enumerate}
        \item Un Utente Autenticato accede alla funzionalitá di Gestione Account
        \item L'Utente Autenticato potrá quindi modificare Password, Email
        \item Le modifiche apportate quindi saranno salvate solo se confermate dall'utente
    \end{enumerate}
    \tableCyan      Scenari Alternativi & 2.1. Una Codemonkey potrà modificare i propri Tag
    \ntableCyan     Requisiti NF        &
    \tableCyan      Punti Aperti        & Se un Utente Registrato cambia la sua email potrebbero esserci incomprensioni?
    \n
\end{tabularx}

%%%%%%%%%%%%%%%%%%%%%%%%%%%%

\begin{tabularx}{\textwidth}
    {|>{\arraybackslash}m{3cm}|>{\arraybackslash}X|}

    \hline \rowcolor{tableCyan!37}
    \large\centering\textbf{Titolo}     & \large\centering\textbf{Imposta Tag di Ricerca}
    \tableCyan      Descrizione         & Gestione dei Tag della Codemonkey
    \ntableCyan     Attori              & Codemonkey
    \tableCyan      Relazioni           & Autenticazione
    \ntableCyan     Precondizioni       &
    \tableCyan      Postcondizioni      & I Tag vengono aggiornati
    \ntableCyan     Scenario Principale &
    \begin{enumerate}
        \item La Codemonkey apre la pagina dedicata alla gestione dei Tag
        \item La Codemonkey aggiunge o rimuove un Tag tra quelli disponibili
        \item La Codemonkey conferma i cambiamenti
    \end{enumerate}
    \tableCyan      Scenari Alternativi &
        2.1. Se il Tag che si vuole aggiungere non é presente tra quelli disponibili, la Codemonkey puó inviare una richiesta per aggiungerlo
    \ntableCyan     Requisiti NF        & I Tag vengono aggiunti digitando in una casella di testo
    \tableCyan      Punti Aperti        & Cosa succede se si tenta di aggiunere un nuovo Tag? Bisognerá che un Amministratore controlli il Tag prima di aggiungerlo?
    \n
\end{tabularx}

%%%%%%%%%%%%%%%%%%%%%%%%%%%%

\begin{tabularx}{\textwidth}
    {|>{\arraybackslash}m{3cm}|>{\arraybackslash}X|}

    \hline \rowcolor{tableCyan!37}
    \large\centering\textbf{Titolo}     & \large\centering\textbf{Segnalazione ad Amministratore}
    \tableCyan      Descrizione         & Funzionalitá per effettuare una Segnalazione ad un Amministratore
    \ntableCyan     Attori              & Codemonkey, Cliente
    \tableCyan      Relazioni           & Lista Collaborazioni
    \ntableCyan     Precondizioni       & Codemonkey e Cliente devono essere autenticati
    \tableCyan      Postcondizioni      & La richiesta sará inoltrata ad un Amministratore
    \ntableCyan     Scenario Principale &
    \begin{enumerate}
        \item La Codemonkey o il Cliente potranno avviare una Segnalazione ad un altro Utente
        \item La Codemonkey o il Cliente compilano un form per motivare la Segnalazione
        \item Viene confermata l'invio della Segnalazione
    \end{enumerate}
    \tableCyan      Scenari Alternativi &
    \ntableCyan     Requisiti NF        &
    \tableCyan      Punti Aperti        &
    \n
\end{tabularx}

%%%%%%%%%%%%%%%%%%%%%%%%%%%%

\begin{tabularx}{\textwidth}
    {|>{\arraybackslash}m{3cm}|>{\arraybackslash}X|}

    \hline  \rowcolor{tableCyan!37}
    \large\centering\textbf{Titolo}     & \large\centering\textbf{Proponi Lavoro}
    \tableCyan      Descrizione         & Sezione dedicata all'invio di una proposta di Lavoro ad una Codemonkey
    \ntableCyan     Attori              & Cliente, Utente
    \tableCyan      Relazioni           & Autenticazione
    \ntableCyan     Precondizioni       & Un Cliente non deve essere sospeso
    \tableCyan      Postcondizioni      & La richiesta di Lavoro viene inoltrata alla Codemonkey
    \ntableCyan     Scenario Principale &
    \begin{enumerate}
        \item Un Utente o un Cliente ha trovato la Codemonkey ideale per sviluppare il suo progetto e puó proporle un Lavoro
        \item L'Utente viene reindirizzato alla sezione di Autenticazione prima di poter iniziare l'inoltro della richiesta
        \item Per mandare la richiesta di Lavoro il Cliente compila un modulo nel quale descrive brevemente l'attività che la Codemonkey svolgerà e i Tag associati a tale attività
        \item Il Cliente conferma di voler contattare la Codemonkey
    \end{enumerate}
    \tableCyan      Scenari Alternativi & 
        2.1 Un Cliente può procedere senza dovere effettuare l'Autenticazione nuovamente. 
        \newline 
        2.2 Un Cliente Sospeso non puó fare alcuna richiesta di Lavoro e verrá reindirizzato ad una pagina di errore
    \ntableCyan     Requisiti NF        & 
    \tableCyan      Punti Aperti        & Per evitare che vengano inviate richieste di Lavoro non sufficentemente dettagliate ci sarà un numero minimo di caratteri da inserire nel campo di testo
    \n
\end{tabularx}

%%%%%%%%%%%%%%%%%%%%%%%%%%%%

\begin{tabularx}{\textwidth}
    {|>{\arraybackslash}m{3cm}|>{\arraybackslash}X|}

    \hline \rowcolor{tableCyan!37}
    \large\centering\textbf{Titolo}     & \large\centering\textbf{Lista Collaborazioni}
    \tableCyan      Descrizione         & Lista di tutte le collaborazioni
    \ntableCyan     Attori              & Codemonkey, Cliente
    \tableCyan      Relazioni           & Gestione Profilo
    \ntableCyan     Precondizioni       & 
    \tableCyan      Postcondizioni      &
    \ntableCyan     Scenario Principale &
    \begin{enumerate}
        \item La Codemonkey o il cliente si autenticano
        \item La Codemonkey o il Cliente dalla Gestione dell'Account potranno accedere alla Lista collaborazioni
        \item Possono visualizzare le collaborazioni attive e passate
    \end{enumerate}
    \tableCyan      Scenari Alternativi &
    \ntableCyan     Requisiti NF        & 
    \tableCyan      Punti Aperti        & Cosa succede se una Codemonkey Interrompe un Lavoro giá iniziato? Il Cliente potrà ancora effettuare una Valutazione?
    \n
\end{tabularx}

%%%%%%%%%%%%%%%%%%%%%%%%%%%%

\begin{tabularx}{\textwidth}
    {|>{\arraybackslash}m{3cm}|>{\arraybackslash}X|}

    \hline  \rowcolor{tableCyan!37}
    \large\centering\textbf{Titolo}     & \large\centering\textbf{Accetta/Rifiuta proposta di Lavoro}
    \tableCyan      Descrizione         & Sezione per la gestione delle Collaborazioni della Codemonkey
    \ntableCyan     Attori              & Codemonkey
    \tableCyan      Relazioni           & Autenticazione
    \ntableCyan     Precondizioni       & Il Lavoro é stato proposto alla Codemonkey
    \tableCyan      Postcondizioni      & Il Lavoro diventa ``in Corso''
    \ntableCyan     Scenario Principale &
    \begin{enumerate}
        \item Alla Codemonkey viene effettuata una proposta di Lavoro da parte di un Cliente
        \item La Codemonkey accetta la proposta
        \item La Codemonkey poi potrá interrompere la richiesta di Lavoro in qualsiasi momento fino a quando non viene terminata da un Cliente
    \end{enumerate}
    \tableCyan      Scenari Alternativi & 2. La Codemonkey rifiuta la proposta
    \ntableCyan     Requisiti NF        &
    \tableCyan      Punti Aperti        &
    \n
\end{tabularx}

%%%%%%%%%%%%%%%%%%%%%%%%%%%%

\begin{tabularx}{\textwidth}
    {|>{\arraybackslash}m{3cm}|>{\arraybackslash}X|}

    \hline  \rowcolor{tableCyan!37}
    \large\centering\textbf{Titolo}     & \large\centering\textbf{Interrompi Lavoro}
    \tableCyan      Descrizione         & Funzionalità per l'interruzione di una Collaborazione in corso da parte della Codemonkey
    \ntableCyan     Attori              & Codemonkey
    \tableCyan      Relazioni           & Autenticazione
    \ntableCyan     Precondizioni       & Il Lavoro é stato già accetato dalla Codemonkey
    \tableCyan      Postcondizioni      & Il Lavoro diventa ``Interrotto''
    \ntableCyan     Scenario Principale &
    \begin{enumerate}
        \item La Codemonkey sceglie un Lavoro che ha in corso per interromperlo
        \item La Codemonkey interrompe il Lavoro e aggiunge un commento sull'interruzione
    \end{enumerate}
    \tableCyan      Scenari Alternativi &
    \ntableCyan     Requisiti NF        & 
    \tableCyan      Punti Aperti        & 
    \n
\end{tabularx}


%%%%%%%%%%%%%%%%%%%%%%%%%%%%

\begin{tabularx}{\textwidth}
    {|>{\arraybackslash}m{3cm}|>{\arraybackslash}X|}

    \hline  \rowcolor{tableCyan!37}
    \large\centering\textbf{Titolo}     & \large\centering\textbf{Termina Collaborazione e Valuta Codemonkey}
    \tableCyan Descrizione              & Sezione per terminare un Lavoro e fornire una Valutazione ad una Codemonkey
    \ntableCyan     Attori              & Cliente
    \tableCyan      Relazioni           & Autenticazione
    \ntableCyan     Precondizioni       &
    \tableCyan      Postcondizioni      & Viene terminato il Lavoro
    \ntableCyan     Scenario Principale &
    \begin{enumerate}
        \item Un Cliente consulta le sue Collaborazioni attive
        \item Il Cliente avvia la procedura per terminare la Collaborazione
        \item Il Cliente Fornisce una valutazione da 0 a 5 alla Codemonkey
        \item Il Cliente puó scrivere quindi una recensione alla Codemonkey (opzionale)
        \item Il Cliente da conferma di voler terminare il Lavoro

    \end{enumerate}
    \tableCyan      Scenari Alternativi & Nel caso il Cliente sia stato Limitato o Sospeso non potrá fornire una valutazione scritta alla Codemonkey
    \ntableCyan     Requisiti NF        &
    \tableCyan      Punti Aperti        & 
    \n
\end{tabularx}


%%%%%%%%%%%%%%%%%%%%%%%%%%%%

\begin{tabularx}{\textwidth}
    {|>{\arraybackslash}m{3cm}|>{\arraybackslash}X|}

    \hline  \rowcolor{tableCyan!37}
    \large\centering\textbf{Titolo}     & \large\centering\textbf{Gestione Sistema}
    \tableCyan      Descrizione         & Sezione dedicata agli Amministratori per la gestione del sistema
    \ntableCyan     Attori              & Amministratore
    \tableCyan      Relazioni           & Autenticazione
    \ntableCyan     Precondizioni       & 
    \tableCyan      Postcondizioni      &
    \ntableCyan     Scenario Principale &
    \begin{enumerate}
        \item L'Amministratore si autentica
        \item L'Amministratore accede alla sezione dedicata alla gestione degli Utenti
        \item L'Amministratore quindi potrá Limitare, Sospendere e Bloccare o Eliminare un Utente Registrato che sceglierà da una lista
        \item L'Amministratore accede alla sezione dedicata alla gestione dei Tag
        \item L'Amministratore quindi puó approvare un Tag proposto o eliminare un Tag esistente
    \end{enumerate}
    \tableCyan      Scenari Alternativi &
    \ntableCyan     Requisiti NF        &
    \tableCyan      Punti Aperti        &
    \n
\end{tabularx}
