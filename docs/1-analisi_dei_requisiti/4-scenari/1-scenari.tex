\newcommand{\tableCyan}{%righe con colori misti inseriti dentro
    \n
    \rowcolor{tableCyan!10}
    \cellcolor{tableCyan!22}
}

\newcommand{\ntableCyan}{%righe con colori misti inseriti dentro
    \n
    \rowcolor{tableCyan!5}
    \cellcolor{tableCyan!12}
}

%%%%%%%%%%%%%%%%%%%%%%%%%%%%

\begin{tabularx}{\textwidth}
    {|m{3cm}|X|}
    \hline \rowcolor{tableCyan!40}
    \large\centering\textbf{Titolo}     & \large \centering\textbf{Registrazione}
    \tableCyan      Descrizione         & Sezione che consente la Registrazione di un Utente
    \ntableCyan     Attori              & Utente
    \tableCyan      Relazioni           &
    \ntableCyan     Precondizioni       &
    \tableCyan      Postcondizioni      & L'Utente viene registrato nel sistema come Codemonkey o come Cliente
    \ntableCyan     Scenario Principale &
    \begin{enumerate}
        \item L'Utente inserisce le credenziali (Username, Email, Password) per la creazione di un Account
        \item L'Utente specifica se vuole essere registrato come Cliente o Codemonkey
        \item L'Utente conferma di volersi registrare
    \end{enumerate}
    \tableCyan      Scenari Alternativi & Se lo Username é giá stato scelto o l'Email immessa é collegata a un Account giá registrato non sará possibile procedere alla Registrazione dell'Utente
    \ntableCyan     Requisiti NF        & Sia nel caso in cui venga inserito un Username giá esistente, sia nel caso venga inserita una Email associata ad un Account esistente, dovrá essere segnalato all'Utente che non é stato possibile effettuare la Registrazione spiegandone la motivazione
    \tableCyan      Punti Aperti        & Verrá abilitata l'autenticazione a 2 fattori giá in fase di Registrazione?\newline
    Bisognerá assicurarsi che l'Utente non si sbagli a digitare la Password, ci sarà un campo di conferma Password?
    \n
\end{tabularx}

\newpage

\begin{tabularx}{\textwidth}
    {|>{\arraybackslash}m{3cm}|>{\arraybackslash}X|}

    \hline \rowcolor{tableCyan!37}
    \large\centering\textbf{Titolo}     & \large\centering\textbf{Autenticazione}
    \tableCyan      Descrizione         & Sezione dedicata all'Autenticazione di un Utente
    \ntableCyan     Attori              & Utente
    \tableCyan      Relazioni           & Gestione Profilo, Proponi Collaborazione, Termina Collaborazione e valuta Codemonkey, Gestione Sistema ,Imposta Tag di Ricerca, Accetta/Rifiuta Collaborazione, Interrompi Collaborazione
    \ntableCyan     Precondizioni       & L'Utente deve essere giá registrato all'interno del sistema oppure deve essere un Amministratore.
    \tableCyan      Postcondizioni      & L'Utente sarà Utente Autenticato oppure Amministratore.
    \ntableCyan     Scenario Principale &
    \begin{enumerate}
        \item L'Utente fornisce le credenziali di accesso al sistema
        \item L'Utente conferma di voler entrare con le credenziali immesse
        \item Nel caso in cui le credenziali siano corrette, Codemonkey o Cliente saranno reindirizzati alla Gestione del Profilo, invece l'Amministratore sará reindirizzato verso la Gestione del Sistema
    \end{enumerate}
    \tableCyan      Scenari Alternativi & Nel caso che i dati forniti per l'autenticazione siano incorretti verrá visualizzato un messaggio di errore\newline Nel caso l'utente sia stato bloccato da un Amministratore verrá fornito un messaggio di errore.
    \ntableCyan     Requisiti NF        & Sará sempre presente un bottone per inoltrare una richiesta di recupero Password all'Amministratore.
    \tableCyan      Punti Aperti        & 
    \n
\end{tabularx}

%%%%%%%%%%%%%%%%%%%%%%%%%%%%

\begin{tabularx}{\textwidth}
    {|>{\arraybackslash}m{3cm}|>{\arraybackslash}X|}

    \hline \rowcolor{tableCyan!37}
    \large\centering\textbf{Titolo}     & \large\centering\textbf{Gestione Profilo}
    \tableCyan      Descrizione         & Sezione dedicata alla gestione dei dati di un Account
    \ntableCyan     Attori              & Cliente, Codemonkey
    \tableCyan      Relazioni           & Autenticazione,Lista Collaborazioni
    \ntableCyan     Precondizioni       & 
    \tableCyan      Postcondizioni      &
    \ntableCyan     Scenario Principale &
    \begin{enumerate}
        \item Un utente si deve autenticare
        \item Un Utente Autenticato accede alla funzionalità di Gestione Account
        \item L'Utente Autenticato potrá quindi modificare Password, Email
        \item Le modifiche apportate quindi saranno salvate solo se confermate dall'utente
    \end{enumerate}
    \tableCyan      Scenari Alternativi & 
    \ntableCyan     Requisiti NF        &
    \tableCyan      Punti Aperti        & Se un Utente Registrato cambia la sua email potrebbero sorgere delle criticità?
    \n
\end{tabularx}

%%%%%%%%%%%%%%%%%%%%%%%%%%%%

\begin{tabularx}{\textwidth}
    {|>{\arraybackslash}m{3cm}|>{\arraybackslash}X|}

    \hline \rowcolor{tableCyan!37}
    \large\centering\textbf{Titolo}     & \large\centering\textbf{Imposta Tag di Ricerca}
    \tableCyan      Descrizione         & Gestione dei Tag della Codemonkey
    \ntableCyan     Attori              & Codemonkey
    \tableCyan      Relazioni           & Autenticazione
    \ntableCyan     Precondizioni       &
    \tableCyan      Postcondizioni      & I Tag vengono aggiornati
    \ntableCyan     Scenario Principale &
    \begin{enumerate}
        \item La Codemonkey si autentica
        \item La Codemonkey apre la pagina dedicata alla gestione dei Tag
        \item La Codemonkey aggiunge o rimuove un Tag tra quelli disponibili
        \item La Codemonkey conferma i cambiamenti
    \end{enumerate}
    \tableCyan      Scenari Alternativi &
        2.1. Se il Tag che si vuole aggiungere non é presente tra quelli disponibili, la Codemonkey puó inviare una richiesta per aggiungerlo
    \ntableCyan     Requisiti NF        & I Tag vengono aggiunti digitando in una casella di testo
    \tableCyan      Punti Aperti        & Cosa succede se si tenta di aggiunere un nuovo Tag? Bisognerá che un Amministratore controlli il Tag prima di aggiungerlo?
    \n
\end{tabularx}

%%%%%%%%%%%%%%%%%%%%%%%%%%%%

\begin{tabularx}{\textwidth}
    {|>{\arraybackslash}m{3cm}|>{\arraybackslash}X|}

    \hline \rowcolor{tableCyan!37}
    \large\centering\textbf{Titolo}     & \large\centering\textbf{Segnalazione ad Amministratore}
    \tableCyan      Descrizione         & Funzionalità per effettuare una Segnalazione ad un Amministratore
    \ntableCyan     Attori              & Codemonkey, Cliente
    \tableCyan      Relazioni           & Lista Collaborazioni
    \ntableCyan     Precondizioni       & 
    \tableCyan      Postcondizioni      & La richiesta sará inoltrata ad un Amministratore
    \ntableCyan     Scenario Principale &
    \begin{enumerate}
        \item La Codemonkey o il cliente consultano la lista Collaborazioni
        \item La Codemonkey o il Cliente potranno quindi avviare una Segnalazione su una qualsiasi Collaborazione
        \item La Codemonkey o il Cliente compilano un form per la motivazione della segnalazione
    \end{enumerate}
    \tableCyan      Scenari Alternativi &
    \ntableCyan     Requisiti NF        &
    \tableCyan      Punti Aperti        & La motivazione dovrà contenere un certo numero di caratteri per essere considerata valida\newline Bisognerà predisporre sistemi che limitino il numero di segnalazioni in arrivo da un utente per evitare abusi del sistema
    \n
\end{tabularx}

%%%%%%%%%%%%%%%%%%%%%%%%%%%%

\begin{tabularx}{\textwidth}
    {|>{\arraybackslash}m{3cm}|>{\arraybackslash}X|}

    \hline  \rowcolor{tableCyan!37}
    \large\centering\textbf{Titolo}     & \large\centering\textbf{Proponi Collaborazione}
    \tableCyan      Descrizione         & Sezione dedicata all'invio di una Collaborazione ad una Codemonkey
    \ntableCyan     Attori              & Cliente, Utente
    \tableCyan      Relazioni           & Autenticazione
    \ntableCyan     Precondizioni       & Un Cliente non deve essere sospeso
    \tableCyan      Postcondizioni      & La richiesta di Collaborazione viene inoltrata alla Codemonkey
    \ntableCyan     Scenario Principale &
    \begin{enumerate}
        \item Un Utente o un Cliente che ha trovato la Codemonkey ideale per sviluppare il suo progetto e puó proporle una Collaborazione
        \item L'Utente viene reindirizzato alla sezione di Autenticazione prima di poter iniziare l'inoltro della richiesta di Collaborazione
        \item Per mandare la richiesta di Collaborazione il Cliente compila un modulo nel quale descrive brevemente l'attività che la Codemonkey svolgerà e i Tag associati a tale attività
        \item Il Cliente conferma di voler contattare la Codemonkey
    \end{enumerate}
    \tableCyan      Scenari Alternativi & 
        2.1 Un Cliente può procedere senza dovere effettuare nuovamente l'Autenticazione
    \ntableCyan     Requisiti NF        & 
    \tableCyan      Punti Aperti        & Per evitare che vengano inviate richieste di Collaborazione non sufficentemente dettagliate ci sarà un numero minimo di caratteri da inserire nel campo di testo
    \n
\end{tabularx}

%%%%%%%%%%%%%%%%%%%%%%%%%%%%

\begin{tabularx}{\textwidth}
    {|>{\arraybackslash}m{3cm}|>{\arraybackslash}X|}

    \hline \rowcolor{tableCyan!37}
    \large\centering\textbf{Titolo}     & \large\centering\textbf{Lista Collaborazioni}
    \tableCyan      Descrizione         & Lista di tutte le collaborazioni
    \ntableCyan     Attori              & Codemonkey, Cliente
    \tableCyan      Relazioni           & Gestione Profilo, Segnala ad Amministratore
    \ntableCyan     Precondizioni       & 
    \tableCyan      Postcondizioni      &
    \ntableCyan     Scenario Principale &
    \begin{enumerate}
        \item La Codemonkey o il Cliente dalla Gestione dell'Account potranno accedere alla Lista collaborazioni
        \item Possono visualizzare e gestire le collaborazioni attive e passate
    \end{enumerate}
    \tableCyan      Scenari Alternativi &
    \ntableCyan     Requisiti NF        & 
    \tableCyan      Punti Aperti        & 
    \n
\end{tabularx}

%%%%%%%%%%%%%%%%%%%%%%%%%%%%

\begin{tabularx}{\textwidth}
    {|>{\arraybackslash}m{3cm}|>{\arraybackslash}X|}

    \hline  \rowcolor{tableCyan!37}
    \large\centering\textbf{Titolo}     & \large\centering\textbf{Accetta/Rifiuta Collaborazione}
    \tableCyan      Descrizione         & Sezione per la gestione delle Collaborazioni della Codemonkey
    \ntableCyan     Attori              & Codemonkey
    \tableCyan      Relazioni           & Autenticazione
    \ntableCyan     Precondizioni       & Il Collaborazione é stata proposta alla Codemonkey
    \tableCyan      Postcondizioni      & Il Collaborazione diventa ``in Corso''
    \ntableCyan     Scenario Principale &
    \begin{enumerate}
        \item La codemonkey si autentica
        \item La Codemonkey sfoglia la sezione dedicata alle nuove Collaborazioni fino a quella di suo interesse
        \item La Codemonkey accetta la Collaborazione
    \end{enumerate}
    \tableCyan      Scenari Alternativi & 3.1 La Codemonkey rifiuta la proposta
    \ntableCyan     Requisiti NF        &
    \tableCyan      Punti Aperti        &La Codemonkey poi potrá interrompere la Collaborazione in qualsiasi momento fino a quando non viene terminata da un Cliente
    \n
\end{tabularx}

%%%%%%%%%%%%%%%%%%%%%%%%%%%%

\begin{tabularx}{\textwidth}
    {|>{\arraybackslash}m{3cm}|>{\arraybackslash}X|}

    \hline  \rowcolor{tableCyan!37}
    \large\centering\textbf{Titolo}     & \large\centering\textbf{Interrompi Collaborazione}
    \tableCyan      Descrizione         & Funzionalità per l'interruzione di una Collaborazione in corso da parte della Codemonkey
    \ntableCyan     Attori              & Codemonkey
    \tableCyan      Relazioni           & Autenticazione
    \ntableCyan     Precondizioni       & Il Collaborazione é stato già accetato dalla Codemonkey
    \tableCyan      Postcondizioni      & Il Collaborazione diventa ``Interrotto''
    \ntableCyan     Scenario Principale &
    \begin{enumerate}
        \item Un utente si autentica come Codemonkey
        \item La Codemonkey sceglie una Collaborazione in corso
        \item La Codemonkey interrompe la Collaborazione e aggiunge un commento sulla motivazione dell'interruzione
    \end{enumerate}
    \tableCyan      Scenari Alternativi &
    \ntableCyan     Requisiti NF        & 
    \tableCyan      Punti Aperti        & Cosa succede se una Codemonkey Interrompe una Collaborazione giá avviata? Il Cliente potrà ancora effettuare una Valutazione?
    \n
\end{tabularx}


%%%%%%%%%%%%%%%%%%%%%%%%%%%%

\begin{tabularx}{\textwidth}
    {|>{\arraybackslash}m{3cm}|>{\arraybackslash}X|}

    \hline  \rowcolor{tableCyan!37}
    \large\centering\textbf{Titolo}     & \large\centering\textbf{Termina Collaborazione e Valuta Codemonkey}
    \tableCyan Descrizione              & Sezione per terminare una Collaborazione e fornire una Valutazione ad una Codemonkey
    \ntableCyan     Attori              & Cliente
    \tableCyan      Relazioni           & Autenticazione
    \ntableCyan     Precondizioni       &
    \tableCyan      Postcondizioni      & Viene terminata la Collaborazione
    \ntableCyan     Scenario Principale &
    \begin{enumerate}
        \item Un cliente si autentica
        \item Un Cliente consulta le sue Collaborazioni attive
        \item Il Cliente avvia la procedura per terminare la Collaborazione
        \item Il Cliente Fornisce una valutazione da 0 a 5 alla Codemonkey
        \item Il Cliente puó scrivere quindi una recensione alla Codemonkey (opzionale)
        \item Il Cliente dà conferma di voler terminare la Collaborazione

    \end{enumerate}
    \tableCyan      Scenari Alternativi & Nel caso il Cliente sia stato Limitato o Sospeso non potrá fornire una valutazione scritta alla Codemonkey
    \ntableCyan     Requisiti NF        &
    \tableCyan      Punti Aperti        & 
    \n
\end{tabularx}


%%%%%%%%%%%%%%%%%%%%%%%%%%%%

\begin{tabularx}{\textwidth}
    {|>{\arraybackslash}m{3cm}|>{\arraybackslash}X|}

    \hline  \rowcolor{tableCyan!37}
    \large\centering\textbf{Titolo}     & \large\centering\textbf{Gestione Sistema}
    \tableCyan      Descrizione         & Sezione dedicata agli Amministratori per la gestione del sistema
    \ntableCyan     Attori              & Amministratore
    \tableCyan      Relazioni           & Autenticazione
    \ntableCyan     Precondizioni       & 
    \tableCyan      Postcondizioni      &
    \ntableCyan     Scenario Principale &
    \begin{enumerate}
        \item L'Amministratore si autentica
        \item L'Amministratore accede alla sezione dedicata alla gestione degli Utenti
        \item L'Amministratore quindi potrá Limitare, Sospendere e Bloccare o Eliminare un Utente Registrato che sceglierà da una lista
        \item L'Amministratore accede alla sezione dedicata alla gestione dei Tag
        \item L'Amministratore quindi puó approvare un Tag proposto o eliminare un Tag esistente
    \end{enumerate}
    \tableCyan      Scenari Alternativi &
    \ntableCyan     Requisiti NF        &
    \tableCyan      Punti Aperti        &
    \n
\end{tabularx}
