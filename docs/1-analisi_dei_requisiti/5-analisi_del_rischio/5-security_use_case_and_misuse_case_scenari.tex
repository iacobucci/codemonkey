\begin{center}%%%%%%%%%%%%%%%
    \rowcolors{2}{orange!20}{orange!5}%colori alternati
    \begin{tabularx}{\textwidth}
        {|>{\raggedright}X|m{5cm}|m{5cm}|}%Opzioni per formato tabella
        \hline
        \rowcolor{orange!50}%settare il colore della riga 
        \textbf{Titolo}                               & \multicolumn{2}{p{10.44cm}|}{\textbf{Disponibilitá}}
        \n  Descrizione                               & \multicolumn{2}{p{10.44cm}|}{Il sistema deve sempre fornire servizio}
        \n  Misuse Case                               & \multicolumn{2}{p{10.44cm}|}{DoS}
        \n  Precondizioni                             & \multicolumn{2}{p{10.44cm}|}{L'attaccante dispone di un ambiente per effettuare un attacco efficace}
        \n  Postcondizioni                            & \multicolumn{2}{p{10.44cm}|}{Il sistema monitora il flusso di dati ed eventualmente attua politiche di protezione o ripristino}
        \n  Scenario Principale                       & Sistema \newline  Registra l'attacco nei Log ed eventualmente esegue un ripristino                                              & Attaccante avvia l'attacco
        \n  Scenario di attacco avvenuto con successo & Sistema \newline  Non riesce a contenere l'attacco \newline Non é possibile il ripristino immediato del sistema                 & Attaccante \newline Riesce a creare un grande flusso di dati \newline Riesce a creare un disservizio
        \n
    \end{tabularx}\label{tab:monkeytable:riskmonke:lianaSicuraOMarcia:Disponibilitá}


    \phantom{M}%%%%%%%%%%%%%%%%%%%%%%%%%%%%%%%%%%%%%%


    \begin{tabularx}{\textwidth}
        {|>{\raggedright}X|m{5cm}|m{5cm}|}%Opzioni per formato tabella
        \hline
        \rowcolor{orange!50}%settare il colore della riga 
        \textbf{Titolo}                               & \multicolumn{2}{p{10.44cm}|}{\textbf{Gatantire protezione}}
        \n  Descrizione                               & \multicolumn{2}{p{10.44cm}|}{Le comunicazioni devono essere protette}
        \n  Misuse Case                               & \multicolumn{2}{p{10.44cm}|}{Man in the middle, sniffing, operazione vietata}
        \n  Precondizioni                             & \multicolumn{2}{p{10.44cm}|}{L'attaccante o il truffatore ha i mezzi per attuare uno sniffing delle comunicazioni e manomettere le operazioni tra il client e il server}
        \n  Postcondizioni                            & \multicolumn{2}{p{10.44cm}|}{Il sistema registra un tentativo di manomissione nei Log}
        \n  Scenario Principale                       & Sistema \newline Cerca di garantire che i dati inviati all'Utente siano protetti, cifrati e non possano essere modificati                                                                                  & Attaccanti \newline - Cercano di intercettare e manomettere le comunicazioni \newline - Cercano di estrarre dati o penetrare nel sistema in modo non autorizzato
        \n  Scenario di attacco avvenuto con successo & Sistema \newline - Garantisce che i dati sensibili siano cifrati in maniera robusta \newline - Cerca di garantire la sicurezza da vulnerabilità                                         & Attaccante \newline - Cerca di aggirare la cifratura per ottenere le informazioni sensibili \newline - Cerca e trova delle vulnerabilità per superare le difese  del sistema
        \n
    \end{tabularx}\label{tab:monkeytable:riskmonke:lianaSicuraOMarciaGarantireProtezione}


    \phantom{M}%%%%%%%%%%%%%%%%%%%%%%%%%%%%%%%%%%%%%%


    \begin{tabularx}{\textwidth}
        {|>{\raggedright}X|m{5cm}|m{5cm}|}%Opzioni per formato tabella
        \hline
        \rowcolor{orange!50}%settare il colore della riga 
        \textbf{Titolo}                               & \multicolumn{2}{p{10.44cm}|}{\textbf{Controllo accesso}}
        \n  Descrizione                               & \multicolumn{2}{p{10.44cm}|}{L'accesso al servizio deve essere granulare e monitorato }
        \n  Misuse Case                               & \multicolumn{2}{p{10.44cm}|}{Furto credenziali, sniffing}
        \n  Precondizioni                             & \multicolumn{2}{p{10.44cm}|}{L'attaccante dispone di un sistema per attuare un attacco a dizionario}
        \n  Postcondizioni                            & \multicolumn{2}{p{10.44cm}|}{Il sistema registra i tentativi ed avverte l'Amministratore}
        \n  Scenario Principale                       & Sistema \newline L'accesso viene negato perché le credenziali sono errate. Il tentativo di accesso viene registrati in un file di Log. \newline Dopo un numero di tentativi di accesso, limitati nel tempo ed errati consecutivamente viene bloccato l'accesso & Attaccante \newline Cerca di individuare ed inserire piu volte le Credenziali di un altro Utente registrato/Amministratore
        \n  Scenario di attacco avvenuto con successo & Sistema \newline Concede l'accesso all'account e ai gruppi a cui l'account é collegato \newline Registra l'accesso nei Log                                                                                              & Attaccante \newline Accede al Sistema \newline Riesce a scaricare dati sensibili
        \n
    \end{tabularx}\label{tab:monkeytable:riskmonke:lianaSicuraOMarciaControlloAccesso}


\end{center}%%%%%%%%%%%%%%%%%%%%%%%