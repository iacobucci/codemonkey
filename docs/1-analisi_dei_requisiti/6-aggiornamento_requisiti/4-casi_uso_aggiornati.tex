
\begin{tabularx}{\textwidth}{|c|X|}
    \hline \rowcolor{tableCyan!37} \large\centering\textbf{Titolo} & \large\centering\textbf{Visualizza Log}
    \tableCyan      Descrizione                                    & Pagina di monitoraggio dei Log
    \ntableCyan     Attori                                         & Amministratore
    \tableCyan      Relazioni                                      & Autenticazione
    \ntableCyan     Precondizioni                                  &
    \tableCyan      Postcondizioni                                 &
    \ntableCyan     Scenario Principale                            &
    \begin{enumerate}
        \item Un Amministratore si autentica
        \item L'Amministratore accede alla sezione dei Log
        \item L'Amministraatore può visualizzare i Log di sistema
        \item L'Amministratore puó scaricare il file di Log
    \end{enumerate}
    \tableCyan      Scenari Alternativi                            &
    \ntableCyan     Requisiti NF                                   &
    \tableCyan      Punti Aperti                                   &
    \n
\end{tabularx}

\begin{tabularx}{\textwidth}
    {|>{\arraybackslash}m{3cm}|>{\arraybackslash}X|}

    \hline \rowcolor{tableCyan!37}
    \large\centering\textbf{Titolo}     & \large\centering\textbf{Autenticazione}
    \tableCyan      Descrizione         & Sezione dedicata all'Autenticazione di un Utente
    \ntableCyan     Attori              & Utente
    \tableCyan      Relazioni           & Gestione Profilo, Proponi Collaborazione, Termina Collaborazione e valuta Codemonkey, Gestione Sistema ,Imposta Tag di Ricerca, Accetta/Rifiuta Collaborazione, Interrompi Collaborazione 
    \ntableCyan     Precondizioni       & L'Utente deve essere giá registrato all'interno del sistema oppure deve essere un Amministratore.
    \tableCyan      Postcondizioni      & L'Utente sarà Utente Autenticato oppure Amministratore.
    \ntableCyan     Scenario Principale &
    \begin{enumerate}
        \item Un utente registrato avrà ricevuto un Qr code per l'autenticazione a due fattori
        \item L'Utente salva il Qr code sul proprio Autenticatore
        \item L'Utente che vole accedere fornisce le credenziali di accesso al sistema e il Totp generato dall'autenticatore   \item L'Utente conferma di voler entrare con le credenziali immesse
        \item Nel caso in cui le credenziali siano corrette, Codemonkey o Cliente saranno reindirizzati alla Gestione del Profilo, invece l'Amministratore sará reindirizzato verso la Gestione del Sistema
    \end{enumerate}
    \tableCyan      Scenari Alternativi & 3.1 Nel caso l'Utente abbia dimenticato la Password oppure non abbia più la possibilità di usare il suo Autenticatore, può chiedere ad un Amministratore la reimpostazione della password e un nuovo Qr code.
     \newline 4.1 Nel caso che i dati forniti per l'autenticazione siano incorretti verrá visualizzato un messaggio di errore\newline Nel caso l'utente sia stato bloccato da un Amministratore verrá fornito un messaggio di errore.
    \ntableCyan     Requisiti NF        & 
    \tableCyan      Punti Aperti        & 
    \n
\end{tabularx}