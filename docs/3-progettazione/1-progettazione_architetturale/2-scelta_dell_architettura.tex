L’applicazione è sviluppata come applicazione web, in modo da essere acces sibile dai dispositivi che supportano le tecnologie web moderne e abbiano a disposizione un web browser.
\begin{itemize}
	\item \large{\textbf{Frontend:}}\newline Il client (web browser) esegue l'applicazione Angular che permette all'utente di interagire con l'applicazione. L'applicazione comunica con il server attraverso richieste HTTP. Angular è un framework Javascript che realizza applicazioni web basate su componenti e che si occupa in automatico dell'ottimizzazione delle prestazioni. In più è stato scelta la libreria Material Design per la stilizzazione dei componenti, in modo da migliorare l'esperienza utente. È disposta anche una API pubblica per richieste REST. 
	\item \large{\textbf{Backend:}}\newline
	L'applicazione lato server è sviluppata utilizzando il framework Rails per Ruby. Ruby è un linguaggio di scripting orientato agli oggetti, che permette di scrivere codice in modo semplice e veloce. Rails è un framework che permette di sviluppare applicazioni web in modo semplice e veloce, seguendo il pattern MVC. Rails si occupa di gestire le richieste HTTP, di gestire l'ORM (object-relational mapping) dei dati e di gestire la comunicazione con il client. Il framework si occupa della maggior parte delle transazioni con il database, offrendo prestazioni e agilità nello sviluppo. Questo servizio è esposto attraverso API REST.
	\item \large{\textbf{Persistenza:}}\newline
	È stato scelto di utilizzare il database PostgreSQL, che è un database relazionale open source, in modo da garantire la persistenza dei dati. Postgresql si integra bene con Rails grazie ai bindings (collegamenti) forniti da librerie esterne.
	Per il salvataggio e il recupero delle immagini è stato scelto di utilizzare un servizio di cloud storage, cioè un server chiamato \textit{Bucket}.
\end{itemize}

È stato adottato il Pattern Broker per la gestione della sessione e per proteggere ulteriormente lo strato dei server. Questa scelta consente di disaccoppiare il client dal backend, fornendo un'interfaccia di comunicazione tramite API REST.\\

Nella configurazione dell'applicazione, il client richiede al server frontend di scaricare l'applicazione Angular. Una volta che l'applicazione è stata scaricata nel browser del client, essa utilizza le API REST per contattare il backend e ottenere i dati necessari per la visualizzazione corretta dell'interfaccia utente.\\

Il server frontend funge da intermediario tra il client e il backend. Riceve le richieste del client, le instrada al backend appropriato e restituisce le risposte al client. In questo modo, il frontend nasconde la struttura dei server backend, consentendo una maggiore flessibilità nel sistema (Dependency inversion principle).\\