\lstset{
  basicstyle=\ttfamily,
  columns=fullflexible,
  frame=single,
  breaklines=true,
  postbreak=\mbox{\textcolor{red}{$\hookrightarrow$}\space},
  showstringspaces=false,
  keywordstyle=\color{blue},
  commentstyle=\color{gray},
  stringstyle=\color{orange}
}

Test di unità con il framework Rails di esempio per il modello \texttt{Collaborazione}, vengono riportate le validazioni che devono essere soddisfatte per poter salvare un'istanza di \texttt{Collaborazione} nel database:

\begin{lstlisting}[language=Ruby]
class Collaborazione < ActiveRecord::Base
  belongs_to :codemonkey
  belongs_to :company
  belongs_to :technologies

  # Database fields
  validates :titolo, presence: true, length: { maximum: 256 }
  validates :descrizione, length: { maximum: 4096 }
  validates :stato, presence: true
  validates :suggest_time, presence: true
  validates :start_time, presence: true
  validates :end_time, presence: true
  validates :valutazione, numericality: { only_integer: true, allow_nil: true, greater_than_or_equal_to: 0, less_than_or_equal_to: 5 } 
  validates :comment, length: { maximum: 1024 }
end
\end{lstlisting}

Test di unità con il framework Angular di esempio per il componente \texttt{UserCardComponent}, vengono riportati i test per la creazione del componente:

\begin{lstlisting}[language=Java]
import { ComponentFixture, TestBed } from '@angular/core/testing';
import { HttpClientTestingModule } from '@angular/common/http/testing';
import { ActivatedRoute } from '@angular/router';
import { of } from 'rxjs';

import { UserComponent } from './user.component';

describe('UserComponent', () => {
  let component: UserComponent;
  let fixture: ComponentFixture<UserComponent>;

  beforeEach(async () => {
    await TestBed.configureTestingModule({
      imports: [HttpClientTestingModule],
      declarations: [UserComponent],
      providers: [
        {
          provide: ActivatedRoute,
          useValue: {
            paramMap: of({ get: () => 'test-username' })
          }
        }
      ]
    }).compileComponents();
  });

  beforeEach(() => {
    fixture = TestBed.createComponent(UserComponent);
    component = fixture.componentInstance;
    fixture.detectChanges();
  });

  it('should create', () => {
    expect(component).toBeTruthy();
  });
});
\end{lstlisting}

Test per i controller con Curl, vengono riportati i test per il controller \texttt{RegistrazioneContoller}:

\begin{lstlisting}[language=bash]
#!/bin/bash

function user/registrazione {
  NAME=$1
  PASSWORD=$2
  EMAIL=$3
  TYPE=$4
  curl -X POST http://localhost:8080/api/user/registrazione -H "Content-Type: application/json" -d "{\"registrazione\": {\"username\": \"$NAME\", \"email\": \"$EMAIL\", \"password\": \"$PASSWORD\", \"type\": \"$TYPE\" }}"
}

# Test case 1: Valid signup
response=$(user/registrazione 'gracehopper' '12345678' 'gracehopper@example.com' 'Codemonkey')
if [[ "$response" == '{"username":"gracehopper","type":"Codemonkey","otp_provisioning_uri":"otpauth://totp/gracehopper?secret='*'"}' ]]; then
  echo "Test case 1: Passed"
else
  echo "Test case 1: Failed"
  exit 1
fi

# Test case 2: Malformed email
response=$(user/registrazione 'edsgerdjikstra' '12345678' 'edsgerdjikstra@malformedemail' 'Codemonkey')
expected_response='{"status":500,"errors":["Malformed Email"]}'
if [[ "$response" == "$expected_response" ]]; then
  echo "Test case 2: Passed"
else
  echo "Test case 2: Failed"
  exit 1
fi
\end{lstlisting}