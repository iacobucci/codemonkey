\documentclass{article}

\usepackage{tocloft}
\usepackage{tabularx}
\usepackage{graphicx}
\usepackage{array}
\usepackage{tabu}
\usepackage[table]{xcolor}

%creazione dei colori custom
\definecolor{tableGreen}{RGB}{25, 200, 60}



%macro per andare a capo nelle tabelle
\newcommand{\n}{
   \\\hline 
}

%macro per generare le tabelle
\newcommand{\monkeytable}[2]{

\begin{center}
\begin{table}[h!]
\setlength{\extrarowheight}{5pt} %padding
\begin{tabularx}{\textwidth}
#2
\n %necessaria per chiudere la tabella
\end{tabularx}
\end{table}
\label{tab:monkeytable}
\end{center}
}




%inizio documento vero e proprio
\title{Codemonkey}

\begin{document}

\begin{figure}[!h]
 \centering
 \includegraphics[width=0.5\linewidth]{images/icon}
 \begin{titlepage}
    \huge Codemonkey
 \end{titlepage}
 \label{fig:nome-etichetta}
\end{figure}

\pagebreak
\tableofcontents
\pagebreak
\input{abstract}
\pagebreak
\section {\Large Raccolta dei requisiti}
\begin{itemize}
\large
\item Gli utenti del servizio Codemonkey si registrano da una pagina e indicano il loro ruolo di programmatore o azienda
\item Il programmatore potrá accedere e modificare i suoi dati e accettare eventuali lavori dopo aver eseguito il login.
\item Una azienda potrá accedere e modificare i suoi dati di presentazione sempre previo login
\item Le aziende possono sfogliare il sito. Per mandare una richiesta di lavoro dovranno essere registrate.
\item I programmatori accetteranno o rifiuteranno il lavoro sempre previo login e una volta terminato il lavoro saranno tenuti a segnare il lavoro come concluso.
\item Le aziende possono dare valutazioni ad ogni programmatore solo se hanno avuto una collaborazione, e la valutazione dovrá contenere il tipo di servizio offerto che verrá utilizzato per una classificazione generale dei programmatori.
\item Programmatori con rating piú alti saranno visualizzati prima nella pagina di esplorazione.
\item Programmatori impegnati in un lavoro avranno minor visibilitá sulla piattaforma.
\item Nelle ricerche saranno presenti filtri per il tipo di lavoro e il budget.
\end{itemize}

\pagebreak
\section {Vocabolario}

\newcommand{\orange}{%onde evitare caos il colore della tabella viene dichiarato qui
    \\
    \rowcolor{orange!20}
    \hline
}
\newcommand{\norange}{
    \\
    \rowcolor{orange!5}
    \hline
}

\monkeytable{3} {
{|>{\centering\arraybackslash}X|>{\raggedright\arraybackslash}m{5cm}|>{\centering\arraybackslash}X|}
%{\opzioni}, m{xcm} per dimensione custom o X per dimensione automatica, elementi racchiusi da ||

\hline %riga in cima alla colonna    
\rowcolor{orange!50}%settare il colore della colonna principale 
\textbf{Voce} &\textbf{Definizione} &\textbf{Sinonimo} %voci della prima colonna

\norange    Programmatore& Utente registrato che fornisce uno o piú servizi alle aziende &
\orange     Azienda& Azienda o un semplice privato interessato a utilizzare uno o piú servizi offerti da un programmatore &
\norange    Credenziali& Metodo di accesso al servizio, basato su username e password &
\orange     Account& Insieme di Credenziali e informazioni che identifica una Azienda/Programmatore&
\norange    Utente & Utilizzatore del servizio (Sia Azienda che  Programmatore) &
\orange     Username & Stringa alfanumericache identifica il nome dell'utente che statentando l'accesso & Identificativo
\norange    Password & Stringa alfanumerica generata da un utente del servizio&
\orange     Autenticazione & Meccanismo di accessocon nome alla piattaforma& Log in
\norange    Registrazione& Funzione di iscrizione alla piattaforma&
}

\pagebreak
\section {Tabelle dei Requisiti}


\newcommand{\tableGreen}{%onde evitare caos il colore della tabella viene dichiarato qui
   \\
   \rowcolor{tableGreen!10}
   \hline
}

\monkeytable{3} {
    {|>{\arraybackslash}m{1.5cm}|>{\centering\arraybackslash}X|}

\hline   
\rowcolor{tableGreen!65}
\multicolumn{2}{|c|}{\textbf{Requisiti Funzionali}}
\n \rowcolor{tableGreen!40} \textbf{ID} & \textbf{Requisito}
\n R1F& un utente qualsiasi può sfogliare il sito e vedere tutti i programmatori 
\tableGreen R2F& il sito deve fornire la possibilità di inserire dei filtri per cercare programmatori con specifiche caratteristiche
\n R3F& è presente una sezione nella quale un utente o una azienda possono autenticarsi
\tableGreen R4F& è presente una sezione nella quale un utente o una azienda possono registrarsi
\n R5F& una azienda potrà contattare un programmatore solo previa autenicazione
\tableGreen R6F& un programmatore può accedere previa autenticazione alla pagina del suo profilo
\n R7F& possono essere effettuate modifiche delle informazioni del programmatore dalla pagina principale
\tableGreen R8F& possono essere accettati i lavori e mandati i preventivi sempre da essa
\n R9F& viene fornita la valutazione di ogni programmatore anche filtrata secondo specifici linguaggi
\tableGreen R10F& viene fornita la possibilità di vedere se un programmatore sta lavorando ad un progetto o meno



}

\monkeytable{3} {
    {|>{\arraybackslash}m{1.5cm}|>{\centering\arraybackslash}X|}

\hline   
\rowcolor{tableGreen!65}
\multicolumn{2}{|c|}{\textbf{Requisiti non Funzionali}}
\n \rowcolor{tableGreen!40} \textbf{ID} & \textbf{Requisito}
\n R1NF& mi butto dalla finestra 
\tableGreen R2NF& rivendo il progetto al miglior offerente e prendo tutti i soldi




}
\pagebreak

\end{document}
