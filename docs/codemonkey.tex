\documentclass{article}

\usepackage{tocloft}
\usepackage{tabularx}
\usepackage{graphicx}
\usepackage{array}
\usepackage{tabu}
\usepackage[table]{xcolor}

%creazione dei colori custom
\definecolor{tableGreen}{RGB}{30, 190, 50}
\definecolor{tableYellow}{RGB}{255, 220, 0}
\definecolor{tableBlue}{RGB}{0 ,20, 210}
\definecolor{tableCyan}{RGB}{0 ,160, 160}
\definecolor{tableRed}{RGB}{220 ,15, 5}

%macro per andare a capo nelle tabelle
\newcommand{\n}{
   \\\hline 
}

%macro per generare le tabelle
\newcommand{\monkeytable}[2]{
\begin{center}
\renewcommand{\arraystretch}{2}%padding top and bottom
\begin{table}[h!]
%\setlength{\extrarowheight}{2pt}%rowheight 
\setlength\tabcolsep{5pt}%padding bordo lat
\begin{tabularx}{\textwidth}
#2
\n %necessaria per chiudere la tabella
\end{tabularx}
\end{table}
\label{tab:monkeytable}
\end{center}
}





%inizio documento vero e proprio
\title{Codemonkey}

\begin{document}

\begin{figure}[!h]
   \centering
   \includegraphics[width=0.5\linewidth]{images/icon}
   \begin{titlepage}
      \huge Codemonkey
   \end{titlepage}
   \label{fig:nome-etichetta}
\end{figure}


\pagebreak
\tableofcontents
\pagebreak
\input{abstract}
\pagebreak
\section {\Large Raccolta dei requisiti}
\begin{itemize}
\large
\item Gli utenti del servizio Codemonkey si registrano da una pagina e indicano il loro ruolo di programmatore o azienda
\item Il programmatore potrá accedere e modificare i suoi dati e accettare eventuali lavori dopo aver eseguito il login.
\item Una azienda potrá accedere e modificare i suoi dati di presentazione sempre previo login
\item Le aziende possono sfogliare il sito. Per mandare una richiesta di lavoro dovranno essere registrate.
\item I programmatori accetteranno o rifiuteranno il lavoro sempre previo login e una volta terminato il lavoro saranno tenuti a segnare il lavoro come concluso.
\item Le aziende possono dare valutazioni ad ogni programmatore solo se hanno avuto una collaborazione, e la valutazione dovrá contenere il tipo di servizio offerto che verrá utilizzato per una classificazione generale dei programmatori.
\item Programmatori con rating piú alti saranno visualizzati prima nella pagina di esplorazione.
\item Programmatori impegnati in un lavoro avranno minor visibilitá sulla piattaforma.
\item Nelle ricerche saranno presenti filtri per il tipo di lavoro e il budget.
\end{itemize}

\pagebreak
\section {Vocabolario}

\newcommand{\orange}{%onde evitare caos il colore della tabella viene dichiarato qui
    \\
    \rowcolor{orange!20}
    \hline
}
\newcommand{\norange}{
    \\
    \rowcolor{orange!5}
    \hline
}

\monkeytable{3} {
{|>{\centering\arraybackslash}X|>{\raggedright\arraybackslash}m{5cm}|>{\centering\arraybackslash}X|}
%{\opzioni}, m{xcm} per dimensione custom o X per dimensione automatica, elementi racchiusi da ||

\hline %riga in cima alla colonna    
\rowcolor{orange!50}%settare il colore della colonna principale 
\textbf{Voce} &\textbf{Definizione} &\textbf{Sinonimo} %voci della prima colonna

\norange    Programmatore& Utente registrato che fornisce uno o piú servizi alle aziende &
\orange     Azienda& Azienda o un semplice privato interessato a utilizzare uno o piú servizi offerti da un programmatore &
\norange    Credenziali& Metodo di accesso al servizio, basato su username e password &
\orange     Account& Insieme di Credenziali e informazioni che identifica una Azienda/Programmatore&
\norange    Utente & Utilizzatore del servizio (Sia Azienda che  Programmatore) &
\orange     Username & Stringa alfanumericache identifica il nome dell'utente che statentando l'accesso & Identificativo
\norange    Password & Stringa alfanumerica generata da un utente del servizio&
\orange     Autenticazione & Meccanismo di accessocon nome alla piattaforma& Log in
\norange    Registrazione& Funzione di iscrizione alla piattaforma&
}

\pagebreak
\section {Tabelle dei Requisiti}


\newcommand{\tableGreen}{%onde evitare caos il colore della tabella viene dichiarato qui
    \\
    \rowcolor{tableGreen!15}
    \hline
}

\newcommand{\ntableGreen}{
    \\
    \rowcolor{tableGreen!5}
    \hline
}

\monkeytable{3} {
{|>{\arraybackslash}m{1.5cm}|>{\centering\arraybackslash}X|}

\hline
\rowcolor{tableGreen!70}
\multicolumn{2}{|c|}{\textbf{Requisiti Funzionali}}
\n \rowcolor{tableGreen!50} \textbf{ID} & \textbf{Requisito}
\ntableGreen    R1F& un utente qualsiasi può sfogliare il sito e vedere tutti i programmatori
\tableGreen     R2F& il sito deve fornire la possibilità di inserire dei filtri per cercare programmatori con specifiche caratteristiche
\ntableGreen    R3F& è presente una sezione nella quale un utente o una azienda possono autenticarsi
\tableGreen     R4F& è presente una sezione nella quale un utente o una azienda possono registrarsi
\ntableGreen    R5F& una azienda potrà contattare un programmatore solo previa autenicazione
\tableGreen     R6F& un programmatore può accedere previa autenticazione alla pagina del suo profilo
\ntableGreen    R7F& possono essere effettuate modifiche delle informazioni del programmatore dalla pagina principale
\tableGreen     R8F& possono essere accettati i lavori e mandati i preventivi sempre da essa
\ntableGreen    R9F& viene fornita la valutazione di ogni programmatore anche filtrata secondo specifici linguaggi
\tableGreen     R10F& viene fornita la possibilità di vedere se un programmatore sta lavorando ad un progetto o meno
}
\newcounter{yellowC}

\newcommand{\Yellow}{%onde evitare caos il colore della tabella viene dichiarato qui
    \\
    \rowcolor{tableYellow!15}
    \hline
    \stepcounter{yellowC}
    R\theyellowC NF
}

\newcommand{\nYellow}{
    \\
    \rowcolor{tableYellow!5}
    \hline
    \stepcounter{yellowC}
    R\theyellowC NF
}
\monkeytable{3} {
{|>{\arraybackslash}m{1.5cm}|>{\arraybackslash}X|}

\hline
\rowcolor{tableYellow!70}
\multicolumn{2}{|c|}{\textbf{Requisiti Non Funzionali}}
\n \rowcolor{tableYellow!50} \textbf{ID} & \centering\textbf{Requisito} \endline
\rowcolor{tableYellow!15}
\hline
\stepcounter{yellowC}
R\theyellowC NF 
            & Il sito deve essere facile da navigare
\nYellow    & Deve essere tracciata l'attivitá dei vari amministratori
\Yellow     & Viene fornita la Valutazione Generale di ogni Codmonkey
\nYellow    & Viene fornita una Valutazione Filtrata del programmatore in base ai filtri impostati nella ricerca del Cliente
\Yellow     & Deve essere possibile vedere a qunti progetti la Codmonkey sta lavorando

}
\newcommand{\tableBlue}{%onde evitare caos il colore della tabella viene dichiarato qui
    \\
    \rowcolor{tableBlue!20}
    \hline
}

\newcommand{\ntableBlue}{
    \\
    \rowcolor{tableBlue!7}
    \hline
}

\monkeytable{3} {
{|>{\arraybackslash}m{1.5cm}|>{\centering\arraybackslash}X|}

\hline
\rowcolor{tableBlue!70}
\multicolumn{2}{|c|}{\textbf{Requisiti di DOminio}}
\n \rowcolor{tableBlue!50} \textbf{ID} & \textbf{Requisito}
\ntableBlue    R1D& 
\tableBlue     R2D& 
\ntableBlue    R3D& 
\tableBlue     R5D& 
\ntableBlue    R6D& 
\tableBlue     R7D& 
}
\pagebreak
\section {Scenari}


\newcommand{\tableCyan}{%onde evitare caos il colore della tabella viene dichiarato qui
    \\
    \hline 
    \rowcolor{tableCyan!20}
    \cellcolor{tableCyan!42.5}
}

\newcommand{\ntableCyan}{
    \\
    \hline
    \rowcolor{tableCyan!12.5}
    \cellcolor{tableCyan!35}
}


\monkeytable{3} {
{|>{\arraybackslash }m{3cm}|>{\centering\arraybackslash}X|}

\hline \rowcolor{tableCyan!70} \textbf{Titolo} & \textbf{...}
\tableCyan     Descrizione&
\ntableCyan    Attori&
\tableCyan     Relazioni&
\ntableCyan    Precondizioni&
\tableCyan     Postcondizioni&
\ntableCyan    Scenario Principale&
\tableCyan     Scenari Alternativi&
\ntableCyan    Requisiti NF&
\tableCyan     Punti Aperti&
}










    % NO    % NO    % NO    % NO    % NO
% NO    % NO    % NO    % NO    % NO

%Si usa per il copia e incolla
%Non  riempitela scimmieeeee                                       monke

\monkeytable{3} {
    {|>{\arraybackslash }m{3cm}|>{\centering\arraybackslash}X|}
    
    \hline \rowcolor{tableCyan!70} \textbf{Titolo} & \textbf{...}
    \tableCyan     Descrizione&
    \ntableCyan    Attori&
    \tableCyan     Relazioni&
    \ntableCyan    Precondizioni&
    \tableCyan     Postcondizioni&
    \ntableCyan    Scenario Principale&
    \tableCyan     Scenari Alternativi&
    \ntableCyan    Requisiti NF&
    \tableCyan     Punti Aperti&
    }

    % NO    % NO    % NO    % NO    % NO
% NO    % NO    % NO    % NO    % NO
                                                                                                                                         %un simpatico easter egg

\pagebreak
\section{Analisi del rischio}

\newcommand{\tableRed}{%onde evitare caos il colore della tabella viene dichiarato qui
    \\
    \rowcolor{tableRed!15}
    \hline
}

\newcommand{\ntableRed}{
    \\
    \rowcolor{tableRed!6}
    \hline
}

\monkeytable{3} {
{|p{2.6cm}|X|X|}
\hline \rowcolor{tableRed!70}  \multicolumn{3}{|c|}{\textbf{Valutazione dei beni}}\\
\hline \rowcolor{tableRed!50} \centering \textbf{Bene} & \centering \textbf{Valore} &  \centering \textbf{Esposizione}\endline
\rowcolor{tableRed!6}
\hline          Credenziali di accesso Codemonkey&Alto:\newline
                Possibilità di modificare le informazioni relative alle Codemonkey.\newline
                Possibilità di rifiutare lavori per conto delle Codemonkey&
                Alta:\newline
                Possibilie perdita economica per la Codemonkey\newline
                Danno di immagine
\tableRed       Credenziali di accesso Clienti&
                Alto:\newline
                Possibilità di proporre lavori fasulli\newline
                Possono essere scritte recensioni false&
                Alta:\newline
                Costi di ripristino\newline
                Possibili spese di rimborso per lavori fasulli già iniziati\newline
                Danno di immmagine

\ntableRed      Credenziali di accesso Amministratori&
                Molto Alto\newline
                Completa gestione di tutti gli Utenti registrati\newline
                Possibilità di vedere lavori non ancora terminati&
                Molto Alta:\newline
                Costi di ripristino di sistema\newline
                Possibile danno di immagine nel caso la notizia diventi di pubblico dominio
\tableRed       DB Utenti Registrati&
                Alto:\newline
                Accesso a tutti i dati degli Utenti registrati&
                lo mettiamo?
                

}

\monkeytable{3} {
{|>{\arraybackslash}p{2.2cm}|>{\arraybackslash}X|>{\arraybackslash}X|>{\arraybackslash}X|}
\hline \rowcolor{tableRed!70}  \multicolumn{4}{|c|}{\textbf{Minacce e Controlli}}\\
\hline \rowcolor{tableRed!50} \centering \textbf{Minaccia} & \centering \textbf{Probabilità} &  \centering \textbf{Controllo} &\centering \textbf{Fattibilità} \endline
\rowcolor{tableRed!6}
\hline      Furto identità Amministratore &
            Molto Bassa\newline
            Username e password stabiliti dall'Amministratore insieme a un sistema per autenticazione a 2 fattori&
            Numero di tentativi disponibili limitato nel tempo\newline
            Autenticazione a 2 fattori che rende valida la sessione corrente\newline
            Log di ogni tentativo di accesso&
            Costo di implementazione Medio-Basso
\tableRed   Furto identità cliente o Codemonkey&
            Media o Bassa\newline
            Username e password scelti in fase di registrazione&
            Numero di tentativi disponibili limitato nel tempo\newline
            Possibilità per un utente registrato di attivare l'autenticazione a 2 fattori\newline
            Possibilità di recuperare l'accouynt tramite mail&
            Costo di implementazione Medio-Basso\newline
\ntableRed  Intercettazione delle comunicazioni&
            Media\newline
            Il servizio è realizzato in rete&
            Utilizzo di un sistema crittografico per la cifratura delle comunicazioni&
            Costo di implementazione Basso\newline
\tableRed   Deny of Service&
            Bassa\newline
            Bassa probabilità di un attacco dos&
            Numero di operazioni di rete possibili limitato nel tempo&
            Basso Costo\newline
            Gestione delle richieste e della rete delegata agli Amministratori


}

\monkeytable{3} {
{|>{\arraybackslash}X|>{\arraybackslash}X|}
\hline \rowcolor{tableRed!70}  \multicolumn{2}{|c|}{\textbf{Tecnologia e Vulnerabilità}}\\
\hline \rowcolor{tableRed!50} \centering \textbf{Tecnologia} & \centering \textbf{Vulnerabilità}  \endline
\rowcolor{tableRed!6}
\hline      Autenticazione & 
            Utente registrato rivela username e password volontariamente o per errore e non ha attivato l'autenticazione a due fattori
\tableRed  Architettura Client/Server &
            Attacco Deny of Service\newline
            Intercettazione delle comunicazioni:\newline
            Man in the middle, Sniffing
}

\end{document}
