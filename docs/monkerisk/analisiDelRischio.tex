\section{Analisi del rischio}


\begin{center}
        \rowcolors{2}{tableRed!15}{tableRed!6}%colori alternati

        \resizebox{1\textwidth}{!}{
                \begin{tabular}{|>\raggedright m{3.5cm}|>\raggedright m{6cm}|>\raggedright m{4.5cm}|}
                        \hline  \rowcolor{tableRed!70}  \multicolumn{3}{|c|}{\Large\textbf{Valutazione dei beni}}
                        \n      \rowcolor{tableRed!50}  \large\textbf{Bene} & \large\textbf{Valore}                                                                                                                                                      & \large\textbf{Esposizione}
                        \n      Credenziali di accesso Codemonkey           & Alto:\newline Possibilità di modificare le informazioni relative alle Codemonkey\newline Possibilità di rifiutare lavori per conto delle Codemonkey                        & Alta:\newline Possibilie perdita economica per la Codemonkey\newline Costi di ripristino\newline Danno di immagine
                        \n      Credenziali di accesso Clienti              & Alto:\newline Possono essere proposti lavori fasulli\newline Possono essere scritte recensioni false                                                                       & Alta:\newline Costi di ripristino\newline Danno di immmagine
                        \n      Credenziali di accesso Amministratori       & Molto Alto:\newline Gestione di tutti gli Utenti Registrati\newline Possibilità visualizzare informazioni sulle segnalazioni e sui lavori non acora terminati degli utenti & Molto Alta:\newline Costi di ripristino di sistema\newline Danno di immagine nel caso la notizia diventi di pubblico dominio
                        \n      DB Utenti Registrati                        & Molto Alto:\newline Accesso a tutti i dati degli Utenti registrati                                                                                                         & Alta:\newline Grave danno di immagine nel caso la notizia diventi di pubblico dominio\newline Costi di ripristino del sistema
                        \n
                \end{tabular}\label{tab:monkeytable:monkerisk:valutaBanane}
        }

        \centering
        \resizebox{1\textwidth}{!}{
                \begin{tabular}{|>\raggedright p{3cm}|>\centering p{2.5cm}|>\raggedright p{5cm}|>\raggedright p{4cm}|}

                        \hline  \rowcolor{tableRed!70}  \multicolumn{4}{|c|}{\Large\textbf{Minacce e Controlli}}
                        \n      \rowcolor{tableRed!50}  \large\textbf{Minaccia} & \large\textbf{Probabilità} & \large\textbf{Controllo}                                                                                                                                             & \large\textbf{Fattibilità}
                        \n      Furto identità Amministratore                   & Molto Bassa                & Numero di tentativi disponibili limitato nel tempo\newline Autenticazione a 2 fattori che rende valida la sessione corrente\newline Log di ogni tentativo di accesso & Costo di implementazione Medio-Basso
                        \n      Furto identità del Cliente o della Codemonkey   & Bassa                      & Numero di tentativi disponibili limitato nel tempo\newline Autenticazione a 2 fattori per il primo accesso da ogni dispositivo                                       & Costo di implementazione Medio-Basso
                        \n      Intercettazione delle comunicazioni             & Media                      & Utilizzo di un sistema crittografico per la cifratura delle comunicazioni                                                                                            & Costo di implementazione Basso
                        \n      Deny of Service                                 & Bassa                      & Numero di operazioni di rete possibili limitato nel tempo                                                                                                            & Costo di implementazione Basso\newline Gestione delle richieste e della rete delegata agli Amministratori
                        \n
                \end{tabular}\label{tab:monkeytable:monkerisk:monkeMinacciataMaMonkeControlla}
        }

        \phantom{M}%%%%%%%%%%%%%%%%%%%%%%%%%%%%%%%%%%%%

        \begin{tabular}{|>\raggedright m{4cm}|>\raggedright m{7.75cm}|}
                
                \hline \rowcolor{tableRed!70}  \multicolumn{2}{|c|}{\Large\textbf{Tecnologia e Vulnerabilità}}
                \n \rowcolor{tableRed!50}  \large\textbf{Tecnologia} & \large\textbf{Vulnerabilità}
                \n  Autenticazione                                   & Un Utente registrato rivela Username e Password volontariamente o per errore e (autenticazione a 2 fattori stuff?)
                \n      Architettura Client/Server                   & Attacco Deny of Service\newline Intercettazione delle comunicazioni:\newline -Man in the middle\newline -Sniffing
                \n
        \end{tabular}\label{tab:monkeytable:monkerisk:monkeVulnerabile}
\end{center}

