\section{Use and Misuse Case scenari}

\begin{center}%%%%%%%%%%%%%%%
    \rowcolors{2}{orange!20}{orange!5}%colori alternati
    \begin{tabularx}{\textwidth}
        {|>\raggedright X|m{5cm}|m{5cm}|}%Opzioni per formato tabella
        \hline
        \rowcolor{orange!50}%settare il colore della riga 
        \textbf{Titolo}                               & \multicolumn{2}{p{10.44cm}|}{\textbf{Disponibilitá}}
        \n  Descrizione                               & \multicolumn{2}{p{10.44cm}|}{Il sistema deve sempre fornire servizio}
        \n  Misuse Case                               & \multicolumn{2}{p{10.44cm}|}{DoS}
        \n  Precondizioni                             & \multicolumn{2}{p{10.44cm}|}{L'attaccante dispone di un ambiente per effettuare un attacco efficace}
        \n  Postcondizioni                            & \multicolumn{2}{p{10.44cm}|}{Il sistema monitora il flusso di dati ed eventualmente attua politiche di protezione o ripristino}
        \n  Scenario Principale                       & Sistema \newline  Registra l'attacco nei log ed eventualmente esegue un ripristino                                              & Attaccante avvia l'attacco
        \n  Scenario di attacco avvenuto con successo & Sistema \newline  Non riesce a contenere l'attacco \newline Non è possibile il ripristino immediato del sistema                 & Attaccante \newline Riesce a creare un grande flusso di dati \newline Riesce a creare un disservizio
        \n
    \end{tabularx}\label{tab:monkeytable:riskmonke:lianaSicuraOMarcia}



    \phantom{M}%%%%%%%%%%%%%%%%%%%%%%%%%%%%%%%%%%%%%%



    \begin{tabularx}{\textwidth}
        {|>\raggedright X|m{5cm}|m{5cm}|}%Opzioni per formato tabella
        \hline
        \rowcolor{orange!50}%settare il colore della riga 
        \textbf{Titolo}                               & \multicolumn{2}{p{10.44cm}|}{\textbf{Gatantire protezione}}
        \n  Descrizione                               & \multicolumn{2}{p{10.44cm}|}{Le comunicazioni e i file devono essere protetti}
        \n  Misuse Case                               & \multicolumn{2}{p{10.44cm}|}{ManInTheMiddle, Sniffing, OperazioneVietata}
        \n  Precondizioni                             & \multicolumn{2}{p{10.44cm}|}{L'attaccante o il truffatore ha i mezzi per attuare uno sniffing delle comunicazioni, manomettere le operazioni tra il client e il server o intaccare la cifratura dei file}
        \n  Postcondizioni                            & \multicolumn{2}{p{10.44cm}|}{Il sistema registra un tentativo di manomissione nei log}
        \n  Scenario Principale                       & Sistema \newline Cerca di garantire che i dati inviati all'Utente siano protetti,cifrati e non possano essere modificati                                                                                  & Attaccanti \newline - Cerca di intercettare e manomettere le comunicazioni \newline - Cerca di estrarre dati o penetrare nel sistema in modo non autorizzato
        \n  Scenario di attacco avvenuto con successo & Sistema \newline - Garantisce che i dati sensibili ed i file salvati siano cifrati in maniera robusta \newline - Cerca di garantire la sicurezza da vulnerabilità                                         & Attaccante \newline - Cerca di aggirare la cifratura per ottenere le informazioni sensibili o i file \newline - Cerca e trova delle vulnerabilità per superare le difese  del sistema
        \n
    \end{tabularx}

    \label{tab:monkeytable:riskmonke:lianaSicuraOMarcia}




    %%%%%%%%%%%%%%%%%%%%%%%%%%%%%%%%%%%%%%%%%%%%%%%%%



    \begin{tabularx}{\textwidth}
        {|>\raggedright X|m{5cm}|m{5cm}|}%Opzioni per formato tabella
        \hline
        \rowcolor{orange!50}%settare il colore della riga 
        \textbf{Titolo}                               & \multicolumn{2}{p{10.44cm}|}{\textbf{Controllo accesso}}
        \n  Descrizione                               & \multicolumn{2}{p{10.44cm}|}{L'accesso al servizio deve essere granulare e monitorato }
        \n  Misuse Case                               & \multicolumn{2}{p{10.44cm}|}{FurtoCredenziali,Sniffing}
        \n  Precondizioni                             & \multicolumn{2}{p{10.44cm}|}{L'attaccante dispone di un sistema per attuare un attacco a dizionario}
        \n  Postcondizioni                            & \multicolumn{2}{p{10.44cm}|}{Il sistema registra i tentativi ed avverte l'Amministratore}
        \n  Scenario Principale                       & Sistema \newline L'accesso viene negato perchè le credenziali sono errate. Il tentativo di accesso viene registrati in un file di log. \newline Dopo 5 tentativi di accesso errati consecutivi viene bloccato l'accesso & Attaccante \newline Cerca di individuare ed inserire piu volte le Credenziali di un altro Utente/Amministratore
        \n  Scenario di attacco avvenuto con successo & Sistema \newline Concede l'accesso all'account e ai gruppi a cui l'account è collegato \newline Registra l'accesso nei log                                                                                              & Attaccante \newline Accede al Sistema \newline Cerca di scaricare tutte le informazioni e di enumerare gli utenti
        \n
    \end{tabularx}\label{tab:monkeytable:riskmonke:lianaSicuraOMarcia}



\end{center}%%%%%%%%%%%%%%%%%%%%%%%



