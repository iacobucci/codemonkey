\section {Tabelle dei Requisiti}

\newcounter{reqFCounter}%counter

\newcommand{\nReqF}{
    \n
    \stepcounter{reqFCounter}
    R\thereqFCounter F
}

\begin{center}
    \rowcolors{2}{tableGreen!5}{tableGreen!15}%colori alternati
    \begin{tabularx}{\textwidth}
        {|c|X|}
        \hline\rowcolor{tableGreen!70}
        \multicolumn{2}{|c|}{\Large\textbf{Requisiti Funzionali}}
        \n \rowcolor{tableGreen!50} \large\textbf{ID} & \large\textbf{Requisito}
        \nReqF & Deve esistere un sistema per registrare nuovi Utenti
        \nReqF & Deve essere disposto un sistema di autenticazione per Utenti giá registrati
        \nReqF & Deve essere possibile eseguire una Ricerca di Codmonkey con l'utilizzo di Filtri
        \nReqF & Una Codmonkey e un Cliente potranno modificare i propri dati da un'area personale
        \nReqF & Una Codmonkey puó accettare o rifiutare una proposta di lavoro presentata da un Cliente
        \nReqF & Una Codmonkey potrá essere visualizzata solo se ha un Profilo Completo
        \nReqF & Un Amministratore deve poter essere in grado di eliminare, bloccare, sospendere e limitare qualsiasi Utente
        \nReqF & Un cliente deve poter effettuare proposte di lavoro ad una Codmonkey
        \nReqF & Un Cliente in stato di sospensione una volta autenticata non potrá piú contattare nuove Codmonkey, potrá solo portare a termine i lavori giá avviati
        \nReqF & Un Cliente limitato potrá solo fornire una valutazione della Codmonkey senza commenti
        \nReqF & Una Codmonkey sospesa non deve piú comparire nelle ricerche effettuate dagli utenti
        \nReqF & Un Utente Registrato quando viene Bloccato termina immediatamente tutte le collaborazioni e non potrá avviarne di nuove
        \n
    \end{tabularx}
\end{center}