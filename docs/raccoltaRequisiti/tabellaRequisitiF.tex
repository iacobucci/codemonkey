\section {Tabelle dei Requisiti}

\newcounter{reqFCounter}%counter

\newcommand{\nReqF}{
    \n
    \stepcounter{reqFCounter}
    R\thereqFCounter F
}

\begin{center}
    \rowcolors{2}{tableGreen!5}{tableGreen!15}%colori alternati
    \begin{tabularx}{\textwidth}
        {|c|X|}
        \hline\rowcolor{tableGreen!70}
        \multicolumn{2}{|c|}{\Large\textbf{Requisiti Funzionali}}
        \n \rowcolor{tableGreen!50}

        \large\textbf{ID}
               & \large\textbf{Requisito}
        \nReqF & Deve esistere un sistema per registrare nuovi Utenti
        \nReqF & Deve essere disposto un sistema di autenticazione per Utenti giá registrati
        \nReqF & Deve essere possibile eseguire una Ricerca di Codmonkey con utilizzo di Filtri
        \nReqF & Una Codmonkey e un Cliente potranno modificare l'email e la password personali
        \nReqF & Una Codmonkey e un Cliente potranno modificare la propria descrizione e contatti
        \nReqF & Una Codmonkey e un Cliente potranno visualizzare tutte le richieste di lavoro attive, da accettare e terminate
        \nReqF & Una Codmonkey deve poter gestire i propri filtri di ricerca
        \nReqF & Un Amministratore deve poter essere in grado di Eliminare, Bloccare, Sospendere e Limitare qualsiasi Utente Registrato
        \nReqF & Un Amministratore deve approvare nuovi filtri di ricerca aggiunti dalle Codmonkey
        \nReqF & Una Codmonkey dovrá avere la possibilitá di accettare o rifiutare una Proposta di Lavoro presentata da un Cliente 
        \nReqF & Una Codmonkey potrá terminare una collaborazione con un Cliente
        \nReqF & Una Codmonkey potrá essere visualizzata solo se ha un Profilo Completo
        \nReqF & Un Cliente puó effettuare proposte di lavoro ad una Codmonkey
        \nReqF & Un Ciente potrá terminare una collaborazione e fornire una valutazione alla Codmonkey
        \n
    \end{tabularx}

    \begin{tabularx}{\textwidth}
        {|c|X|}
        \hline\rowcolor{tableGreen!70}
        \multicolumn{2}{|c|}{\Large\textbf{Requisiti Funzionali}}
        \n \rowcolor{tableGreen!50}

        \large\textbf{ID}
               & \large\textbf{Requisito}
        \nReqF & Un Cliente in stato di sospensione una volta autenticata non potrá piú contattare nuove Codmonkey, potrá solo portare a termine i lavori giá avviati
        \nReqF & Una Codmonkey sospesa non deve piú comparire nelle ricerche effettuate dagli Utenti
        \nReqF & Un Utente Registrato quando viene Bloccato termina immediatamente tutte le collaborazioni e non potrá piú avviarne di nuove
        \nReqF & Un Utente Registrato potrá avviare una segnalazione su un qualunque Lavoro sia che sia cliente sia che sia Codmonkey
        \n
    \end{tabularx}
\end{center}