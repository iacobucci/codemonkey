\newcounter{reqNFCounter} %counter

\newcommand{\nReqNF}{
    \n
    \stepcounter{reqNFCounter}
    R\thereqNFCounter NF
}

\begin{center}
    \rowcolors{2}{tableYellow!5}{tableYellow!10} %colori alternati
    \begin{tabularx}{\textwidth}
        {|c|X|}
        \hline\rowcolor{tableYellow!60}
        \multicolumn{2}{|c|}{\Large\textbf{Requisiti Non Funzionali}}
        \n \rowcolor{tableYellow!35} \large\textbf{ID}
                & \large\textbf{Requisito}
        \nReqNF & Il sito deve essere facile da navigare
        \nReqNF & Si potrá vedere quanti lavori la Codmonkey sta attualmente svolgendo
        \nReqNF & Sará possibile vedere a quanti lavori la Codmonkey ha preso parte
        \nReqNF & Un Utente potrá vedere 2 valutazioni della Codmonkey: una valutazione generale di tutti i lavori svolti e una che rispecchi solo ció che é stato fatto nell'ambito dei filtri selezionati dal Cliente
        \n
    \end{tabularx}
\end{center}