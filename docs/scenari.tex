\section {Scenari}


\newcommand{\tableCyan}{%onde evitare caos il colore della tabella viene dichiarato qui
    \\
    \hline 
    \rowcolor{tableCyan!20}
    \cellcolor{tableCyan!42.5}
}

\newcommand{\ntableCyan}{
    \\
    \hline
    \rowcolor{tableCyan!12.5}
    \cellcolor{tableCyan!35}
}


\monkeytable{3} {
{|>{\arraybackslash }m{3cm}|>{\centering\arraybackslash}X|}

\hline \rowcolor{tableCyan!70} \textbf{Titolo} & \textbf{...}
\tableCyan     Descrizione&
\ntableCyan    Attori&
\tableCyan     Relazioni&
\ntableCyan    Precondizioni&
\tableCyan     Postcondizioni&
\ntableCyan    Scenario Principale&
\tableCyan     Scenari Alternativi&
\ntableCyan    Requisiti NF&
\tableCyan     Punti Aperti&
}










    % NO    % NO    % NO    % NO    % NO
% NO    % NO    % NO    % NO    % NO

%Si usa per il copia e incolla
%Non  riempitela scimmieeeee                                       monke

\monkeytable{3} {
    {|>{\arraybackslash }m{3cm}|>{\centering\arraybackslash}X|}
    
    \hline \rowcolor{tableCyan!70} \textbf{Titolo} & \textbf{...}
    \tableCyan     Descrizione&
    \ntableCyan    Attori&
    \tableCyan     Relazioni&
    \ntableCyan    Precondizioni&
    \tableCyan     Postcondizioni&
    \ntableCyan    Scenario Principale&
    \tableCyan     Scenari Alternativi&
    \ntableCyan    Requisiti NF&
    \tableCyan     Punti Aperti&
    }

    % NO    % NO    % NO    % NO    % NO
% NO    % NO    % NO    % NO    % NO
                                                                                                                                         %un simpatico easter egg
