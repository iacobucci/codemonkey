\section {Scenari}


\newcommand{\tableCyan}{%onde evitare caos il colore della tabella viene dichiarato qui
    \\
    \hline 
    \rowcolor{tableCyan!10}
    \cellcolor{tableCyan!22}
}

\newcommand{\ntableCyan}{
    \\
    \hline
    \rowcolor{tableCyan!5}
    \cellcolor{tableCyan!12}
}


\monkeytable{3} {
{|>{\arraybackslash }m{3cm}|>{\arraybackslash}X|}

\hline \rowcolor{tableCyan!37} \centering\textbf{Titolo} & \centering\textbf{...}\endline
\hline 
\rowcolor{tableCyan!10}
\cellcolor{tableCyan!22}

               Descrizione &test
\ntableCyan    Attori &
\tableCyan     Relazioni &
\ntableCyan    Precondizioni &
\tableCyan     Postcondizioni &
\ntableCyan    Scenario Principale &
\tableCyan     Scenari Alternativi &
\ntableCyan    Requisiti NF &
\tableCyan     Punti Aperti &
}










    % NO    % NO    % NO    % NO    % NO
% NO    % NO    % NO    % NO    % NO

%Si usa per il copia e incolla
%Non  riempitela scimmieeeee                                       monke

\monkeytable{3} {
{|>{\arraybackslash }m{3cm}|>{\arraybackslash}X|}

\hline \rowcolor{tableCyan!37} \centering\textbf{Titolo} & \centering\textbf{...}\endline
\hline 
\rowcolor{tableCyan!10}
\cellcolor{tableCyan!22}

               Descrizione &test
\ntableCyan    Attori &
\tableCyan     Relazioni &
\ntableCyan    Precondizioni &
\tableCyan     Postcondizioni &
\ntableCyan    Scenario Principale &
\tableCyan     Scenari Alternativi &
\ntableCyan    Requisiti NF &
\tableCyan     Punti Aperti &
}

    % NO    % NO    % NO    % NO    % NO
% NO    % NO    % NO    % NO    % NO
                                                                                                                                         %un simpatico easter egg
