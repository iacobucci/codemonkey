\hypersetup{
    colorlinks=true,
    linkcolor=black,
    filecolor=magenta,
    urlcolor=cyan,
}

\lstset{
    basicstyle=\ttfamily\small,
    breaklines=true,
    frame=single,
    keywordstyle=\color{blue},
    stringstyle=\color{red},
    commentstyle=\color{green!70!black},
    showstringspaces=false,
    numbers=left,
    numberstyle=\tiny,
    numbersep=8pt,
    stepnumber=1,
    tabsize=4,
}

%creazione dei colori custom
\definecolor{tableGreen}{RGB}{25, 175, 45}
\definecolor{tableYellow}{RGB}{255, 220, 0}
\definecolor{tableBlue}{RGB}{0 ,20, 210}
\definecolor{tableCyan}{RGB}{0 ,160, 156}
\definecolor{tableRed}{RGB}{235 ,15, 0}

%macro per andare a capo nelle tabelle
\newcommand{\n}{
    \tabularnewline\hline
}

%robe per tabelle
\renewcommand{\arraystretch}{2.4}%padding sopra sotto
\setlength\tabcolsep{6pt}%padding bordo lat


%macro per generare le tabelle
\newcommand{\monkeytable}[2]{
    \begin{center}
        \begin{tabularx}{\textwidth}
            #2
            \n %necessaria per chiudere la tabella
        \end{tabularx}
    \end{center}
    \label{tab:monkeytable}
}

\usetikzlibrary{positioning}
\usetikzlibrary{arrows.meta}
\usetikzlibrary{shapes.multipart}
\usetikzlibrary{calc}
\tikzset{inheritance line/.style={->, >=open diamond, thick, black}}

\newcommand{\subsubsubsection}[1]{\paragraph{#1}\mbox{}\\}
% Modifica del contatore di profondità
\setcounter{secnumdepth}{4}
\setcounter{tocdepth}{4}

