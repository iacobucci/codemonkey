\section {Tabelle dei Requisiti}


\newcommand{\tableGreen}{%onde evitare caos il colore della tabella viene dichiarato qui
   \\
   \rowcolor{tableGreen!10}
   \hline
}

\monkeytable{3} {
    {|>{\arraybackslash}m{1.5cm}|>{\centering\arraybackslash}X|}

\hline   
\rowcolor{tableGreen!65}
\multicolumn{2}{|c|}{\textbf{Requisiti Funzionali}}
\n \rowcolor{tableGreen!40} \textbf{ID} & \textbf{Requisito}
\n R1F& un utente qualsiasi può sfogliare il sito e vedere tutti i programmatori 
\tableGreen R2F& il sito deve fornire la possibilità di inserire dei filtri per cercare programmatori con specifiche caratteristiche
\n R3F& è presente una sezione nella quale un utente o una azienda possono autenticarsi
\tableGreen R4F& è presente una sezione nella quale un utente o una azienda possono registrarsi
\n R5F& una azienda potrà contattare un programmatore solo previa autenicazione
\tableGreen R6F& un programmatore può accedere previa autenticazione alla pagina del suo profilo
\n R7F& possono essere effettuate modifiche delle informazioni del programmatore dalla pagina principale
\tableGreen R8F& possono essere accettati i lavori e mandati i preventivi sempre da essa
\n R9F& viene fornita la valutazione di ogni programmatore anche filtrata secondo specifici linguaggi
\tableGreen R10F& viene fornita la possibilità di vedere se un programmatore sta lavorando ad un progetto o meno



}

\monkeytable{3} {
    {|>{\arraybackslash}m{1.5cm}|>{\centering\arraybackslash}X|}

\hline   
\rowcolor{tableGreen!65}
\multicolumn{2}{|c|}{\textbf{Requisiti non Funzionali}}
\n \rowcolor{tableGreen!40} \textbf{ID} & \textbf{Requisito}
\n R1NF& mi butto dalla finestra 
\tableGreen R2NF& rivendo il progetto al miglior offerente e prendo tutti i soldi




}