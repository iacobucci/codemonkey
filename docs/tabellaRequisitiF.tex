\section {Tabelle dei Requisiti}


\newcommand{\tableGreen}{%onde evitare caos il colore della tabella viene dichiarato qui
    \\
    \rowcolor{tableGreen!15}
    \hline
}

\newcommand{\ntableGreen}{
    \\
    \rowcolor{tableGreen!5}
    \hline
}

\monkeytable{3} {
{|>{\arraybackslash}m{1.5cm}|>{\centering\arraybackslash}X|}

\hline
\rowcolor{tableGreen!70}
\multicolumn{2}{|c|}{\textbf{Requisiti Funzionali}}
\n \rowcolor{tableGreen!50} \textbf{ID} & \textbf{Requisito}
\ntableGreen    R1F& un utente qualsiasi può sfogliare il sito e vedere tutti i programmatori
\tableGreen     R2F& il sito deve fornire la possibilità di inserire dei filtri per cercare programmatori con specifiche caratteristiche
\ntableGreen    R3F& è presente una sezione nella quale un utente o una azienda possono autenticarsi
\tableGreen     R4F& è presente una sezione nella quale un utente o una azienda possono registrarsi
\ntableGreen    R5F& una azienda potrà contattare un programmatore solo previa autenicazione
\tableGreen     R6F& un programmatore può accedere previa autenticazione alla pagina del suo profilo
\ntableGreen    R7F& possono essere effettuate modifiche delle informazioni del programmatore dalla pagina principale
\tableGreen     R8F& possono essere accettati i lavori e mandati i preventivi sempre da essa
\ntableGreen    R9F& viene fornita la valutazione di ogni programmatore anche filtrata secondo specifici linguaggi
\tableGreen     R10F& viene fornita la possibilità di vedere se un programmatore sta lavorando ad un progetto o meno
}