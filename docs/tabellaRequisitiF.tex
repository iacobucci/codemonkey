\section {Tabelle dei Requisiti}

\newcounter{greenC}

\newcommand{\tableGreen}{%onde evitare caos il colore della tabella viene dichiarato qui
    \\
    \rowcolor{tableGreen!15}
    \hline
    \stepcounter{greenC}
    R\thegreenC F
}

\newcommand{\ntableGreen}{
    \\
    \rowcolor{tableGreen!5}
    \hline
    \stepcounter{greenC}
    R\thegreenC F
}

\monkeytable{3} {
{|>{\arraybackslash}m{1.5cm}|>{\arraybackslash}X|}

\hline
\rowcolor{tableGreen!70}
\multicolumn{2}{|c|}{\textbf{Requisiti Funzionali}}
\n \rowcolor{tableGreen!50} \textbf{ID} &\centering \textbf{Requisito} \endline
\rowcolor{tableGreen!5}
    \hline
    \stepcounter{greenC}
    R\thegreenC F
                & un utente qualsiasi può sfogliare il sito e vedere tutti i programmatori
\tableGreen     & il sito deve fornire la possibilità di inserire dei filtri per cercare programmatori con specifiche caratteristiche
\ntableGreen    & è presente una sezione nella quale un utente o una azienda possono autenticarsi
\tableGreen     & è presente una sezione nella quale un utente o una azienda possono registrarsi
\ntableGreen    & una azienda potrà contattare un programmatore solo previa autenicazione
\tableGreen     & un programmatore può accedere previa autenticazione alla pagina del suo profilo
\ntableGreen    & possono essere effettuate modifiche delle informazioni del programmatore dalla pagina principale
\tableGreen     & possono essere accettati i lavori e mandati i preventivi sempre da essa
\ntableGreen    & viene fornita la valutazione di ogni programmatore anche filtrata secondo specifici linguaggi
\tableGreen     & viene fornita la possibilità di vedere se un programmatore sta lavorando ad un progetto o meno
}