\newcounter{redC}

\newcommand{\tableRed}{%onde evitare caos il colore della tabella viene dichiarato qui
    \\
    \rowcolor{tableRed!15}
    \hline
    \stepcounter{redC}
    R\theredC NF
}

\newcommand{\ntableRed}{
    \\
    \rowcolor{tableRed!5}
    \hline
    \stepcounter{redC}
    R\theredC NF
}
\monkeytable{3} {
{|>{\arraybackslash}m{1.5cm}|>{\centering\arraybackslash}X|}

\hline
\rowcolor{tableRed!70}
\multicolumn{2}{|c|}{\textbf{Requisiti Non Funzionali}}
\n \rowcolor{tableRed!50} \textbf{ID} & \textbf{Requisito}
\ntableRed    &
\tableRed     &
\ntableRed    &
\tableRed     &
\ntableRed    &
\tableRed     &
}