\newcounter{yellowC}

\newcommand{\Yellow}{%onde evitare caos il colore della tabella viene dichiarato qui
    \\
    \rowcolor{tableYellow!15}
    \hline
    \stepcounter{yellowC}
    R\theyellowC NF
}

\newcommand{\nYellow}{
    \\
    \rowcolor{tableYellow!5}
    \hline
    \stepcounter{yellowC}
    R\theyellowC NF
}
\monkeytable{3} {
{|>{\arraybackslash}m{1.5cm}|>{\arraybackslash}X|}

\hline
\rowcolor{tableYellow!70}
\multicolumn{2}{|c|}{\textbf{Requisiti Non Funzionali}}
\n \rowcolor{tableYellow!50} \textbf{ID} & \centering\textbf{Requisito} \endline
\rowcolor{tableYellow!15}
\hline
\stepcounter{yellowC}
R\theyellowC NF 
            & Il sito deve essere facile da navigare
\nYellow    & Deve essere tracciata l'attivitá dei vari amministratori
\Yellow     & Viene fornita la Valutazione Generale di ogni Codmonkey
\nYellow    & Viene fornita una Valutazione Filtrata del programmatore in base ai filtri impostati nella ricerca del Cliente
\Yellow     & Deve essere possibile vedere a qunti progetti la Codmonkey sta lavorando

}