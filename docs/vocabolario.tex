\section {Vocabolario}

\newcommand{\orange}{%onde evitare caos il colore della tabella viene dichiarato qui
   \\
   \rowcolor{orange!10}
   \hline
}

\monkeytable{3} {
    {|>{\centering\arraybackslash}X|>{\raggedright\arraybackslash}m{5cm}|>{\centering\arraybackslash}X|}
%{\opzioni}, m{xcm} per dimensione custom o X per dimensione automatica, elementi racchiusi da ||

\hline %riga in cima alla colonna    
\rowcolor{orange!45}%settare il colore della colonna principale 
\textbf{Voce} &\textbf{Definizione} &\textbf{Sinonimo} %voci della prima colonna
    
    \n      Programmatore& Utente registrato che fornisce uno o piú servizi alle aziende & 
    \orange Azienda& Azienda o un semplice privato interessato a utilizzare uno o piú servizi offerti da un programmatore &
    \n      Credenziali& Metodo di accesso al servizio, basato su username e password &
    \orange Account& Insieme di Credenziali e informazioni che identifica una Azienda/Programmatore& 
    \n      Utente & Utilizzatore del servizio (Sia Azienda che  Programmatore) &
    \orange Username & Stringa alfanumericache identifica il nome dell'utente che statentando l'accesso & Identificativo 
    \n      Password & Stringa alfanumerica generata da un utente del servizio&
    \orange Autenticazione & Meccanismo di accessocon nome alla piattaforma& Log in
    \n      Registrazione& Funzione di iscrizione alla piattaforma&
}
