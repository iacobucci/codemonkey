% per le tabelle abbiamo la macro /monkeytable
% la macro prende come arogmento la dimensione della tabella e il contenuto della tabella. Se la prima riga indica il nome delle colonne, bisogna mettere \textbf{nome_colonna} per renderla in grassetto


%tabella Scenari:

\monkeytable{3} {
{|>{\arraybackslash }m{3cm}|>{\centering\arraybackslash}X|}

\hline \rowcolor{tableCyan!50} \textbf{Titolo} & \textbf{...}
\tableCyan     Descrizione&
\ntableCyan    Attori&
\tableCyan     Relazioni&
\ntableCyan    Precondizioni&
\tableCyan     Postcondizioni&
\ntableCyan    Scenario Principale&
\tableCyan     Scenari Alternativi&
\ntableCyan    Requisiti NF&
\tableCyan     Punti Aperti&
}
